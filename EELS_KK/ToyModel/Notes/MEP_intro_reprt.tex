\documentclass[12pt,a4paper]{article}
%\documentclass[11pt]{iopart}

\usepackage[colorlinks=true, linkcolor=black!50!blue, urlcolor=blue, citecolor=blue, anchorcolor=blue]{hyperref}
\usepackage[font=small,labelfont=bf,margin=0mm,labelsep=period,tableposition=top]{caption}
\usepackage[a4paper,top=3cm,bottom=2.5cm,left=2.5cm,right=2.5cm,bindingoffset=0mm]{geometry}

\usepackage{graphicx}
\usepackage{float}
\usepackage{afterpage}
\usepackage{epsfig}%,cite}
\usepackage{amssymb}
\usepackage{amsmath}
\usepackage{bm}
%\usepackage{dsfont}
\usepackage{multirow}
\usepackage{url}
\usepackage{xcolor}
\usepackage{float}
\usepackage{afterpage}
\usepackage{ulem}

\usepackage{url}
\usepackage{hyperref}

\usepackage{multirow,booktabs,multirow}

%\bibliographystyle{iopart-num}
%\usepackage{natbib}
%\bibliographystyle{JHEP}
%\bibliographystyle{ieee}


\usepackage[ backend=bibtex, style=ieee, natbib=true]{biblatex}
%\usepackage[backend=bibtex,style=ieee-alphabetic,natbib=true]{biblatex}
\addbibresource{bib}



%%%%%%%%%%%%%%%%%%%%%%%%%%%%%%%%%%%%%%%%%%%%%%%%%%%%%%%%%%%%%

\def\smallfrac#1#2{\hbox{$\frac{#1}{#2}$}}
\newcommand{\be}{\begin{equation}}
\newcommand{\ee}{\end{equation}}
\newcommand{\bea}{\begin{eqnarray}}
\newcommand{\eea}{\end{eqnarray}}
\newcommand{\ei}{\end{itemize}}
\newcommand{\ben}{\begin{enumerate}}
\newcommand{\een}{\end{enumerate}}
\newcommand{\la}{\left\langle}
\newcommand{\ra}{\right\rangle}
\newcommand{\lc}{\left[}
  \newcommand{\tr}{\toprule}
  \newcommand{\mr}{\midrule}
  \newcommand{\br}{\bottomrule}
\newcommand{\rc}{\right]}
\newcommand{\lp}{\left(}
\newcommand{\rp}{\right)}
\newcommand{\as}{\alpha_s}
\newcommand{\aq}{\alpha_s\left( Q^2 \right)}
\newcommand{\amz}{\alpha_s\left( M_Z^2 \right)}
\newcommand{\aqq}{\alpha_s \left( Q^2_0 \right)}
\newcommand{\aqz}{\alpha_s \left( Q^2_0 \right)}
\def\toinf#1{\mathrel{\mathop{\sim}\limits_{\scriptscriptstyle
{#1\rightarrow\infty }}}}
\def\tozero#1{\mathrel{\mathop{\sim}\limits_{\scriptscriptstyle
{#1\rightarrow0 }}}}
\def\toone#1{\mathrel{\mathop{\sim}\limits_{\scriptscriptstyle
{#1\rightarrow1 }}}}
\def\frac#1#2{{{#1}\over {#2}}}
\def\gsim{\mathrel{\rlap{\lower4pt\hbox{\hskip1pt$\sim$}}
    \raise1pt\hbox{$>$}}}       
\def\lsim{\mathrel{\rlap{\lower4pt\hbox{\hskip1pt$\sim$}}
    \raise1pt\hbox{$<$}}}       
\newcommand{\mrexp}{\mathrm{exp}}
\newcommand{\dat}{\mathrm{dat}}
\newcommand{\one}{\mathrm{(1)}}
\newcommand{\two}{\mathrm{(2)}}
\newcommand{\art}{\mathrm{art}}
\newcommand{\rep}{\mathrm{rep}}
\newcommand{\net}{\mathrm{net}}
\newcommand{\stopp}{\mathrm{stop}}
\newcommand{\sys}{\mathrm{sys}}
\newcommand{\stat}{\mathrm{stat}}
\newcommand{\diag}{\mathrm{diag}}
\newcommand{\pdf}{\mathrm{pdf}}
\newcommand{\tot}{\mathrm{tot}}
\newcommand{\minn}{\mathrm{min}}
\newcommand{\mut}{\mathrm{mut}}
\newcommand{\partt}{\mathrm{part}}
\newcommand{\dof}{\mathrm{dof}}
\newcommand{\NS}{\mathrm{NS}}
\newcommand{\cov}{\mathrm{cov}}
\newcommand{\gen}{\mathrm{gen}}
\newcommand{\cut}{\mathrm{cut}}
\newcommand{\parr}{\mathrm{par}}
\newcommand{\val}{\mathrm{val}}
\newcommand{\reff}{\mathrm{ref}}
\newcommand{\Mll}{M_{ll}}
\newcommand{\extra}{\mathrm{extra}}
\newcommand{\draft}[1]{}
% Added by MU 
\def \a{\alpha}
\def \b{\beta}
\def \g{\gamma}
\def \z{\zeta}
\def \t{{\bf T}} % vector of theoretical predictions
\def \c{{\bf c}} % vector of coefficients of theoretical predictions
\def \y{{\bf y}} % vector of experimental data
\def \s{{\bf \sigma}} % experimental covariance matrix
% Added by JR
\def\lapprox{\lower .7ex\hbox{$\;\stackrel{\textstyle <}{\sim}\;$}}
\def\gapprox{\lower .7ex\hbox{$\;\stackrel{\textstyle >}{\sim}\;$}}
\def\half{\smallfrac{1}{2}}
\def\GeV{{\rm GeV}}
\def\TeV{{\rm TeV}}
\def\ap{{a'}}
\def\vp{{v'}}
\def\e{\epsilon}
\def\d{{\rm d}}
\def\calN{{\cal N}}
\def\shat{\hat{s}}
\def\barq{\bar{q}}
\def\qq{q \bar q}
\def\uu{u \bar u}
\def\dd{d \bar d}
\def\pp{p \bar p}
\def\xa{x_{1}}
\def\xb{x_{2}}
\def\xaa{x_{1}^{0}}
\def\xbb{x_{2}^{0}}
\def\smx{\stackrel{x\to 0}{\longrightarrow}}
\def\Li{{\rm Li}}
\numberwithin{equation}{section}
\numberwithin{figure}{section}
\numberwithin{table}{section}
\newcommand{\tmop}[1]{\ensuremath{\operatorname{#1}}}
\newcommand{\tmtextit}[1]{{\itshape{#1}}}
\newcommand{\tmtextrm}[1]{{\rmfamily{#1}}}
\newcommand{\tmtexttt}[1]{{\ttfamily{#1}}}
\usepackage{tabularx}
\newcolumntype{C}[1]{>{\centering\arraybackslash}p{#1}}
\begin{document}
\newgeometry{top=1.5cm,bottom=1.5cm,left=2.5cm,right=2.5cm,bindingoffset=0mm}




\begin{center}
  {\Large \bf MEP Introduction Report}
\vspace{1.4cm}

Isabel Postmes

\vspace{1.0cm}
 
{\it \small

$^{1}$Kavli Institute of Nanoscience, Delft University of Technology, 2628CJ Delft, The
  Netherlands\\[0.1cm]

}

\vspace{1.0cm}

{\bf \large Abstract}

\end{center}

MEP Introduction Report on the master thesis project of Isabel Postmes, supervised by Sonia Conesa Boj.
My Master thesis will revolve around the automization of the electron energy loss spectrum (EELS) analysis, documented by the transmission electron microscope (TEM). The focus will be on the low-loss spectrum, where the influence of the zero loss peak (ZLP) is not negligible, and will include an error estimation, which makes it unique. The first step in this was done by Laurien Roest: the detection of the ZLP. I will focuss in my months in generilizing the program Laurien wrote, and I will expand it by including the Kramer-Kronig analysis and peak detection. The final goal is to be able to input an image of spectra, and have an automatic output of what pixels statisfy certain criteria. 



\clearpage
\tableofcontents

\section{Planning}
The first couple of weeks will be dedicated to getting acquainted with the subjects at hand, and Lauriens program. Then I will build the program step-by-step: starting by implementing the Fourier Deconvolution, continuing to subtracting the dielectric function with the Kramer-Kronig analysis and than working on the peak detection. For each step I will start by working out a toy model, which is a simplified idealized spectrum. The expected results of such a spectrum can be worked out analytically, which creates the opportunity to validate the functions. Eventually I will implement all this into a program which takes a image of spectra, and gives a visual output and list of pixels depending on what characteristics are selected. 

To test the final program, we will measure layered materials.

\newpage
\subsection{Milestones}
Even though I find it hard to predict the exact time I will be needing for each step, since with programming problems some expected hurdles can be overcome with relative ease, whereas other smaller tasks can prove an extensive challenge. However, as for now, I would be happy if my time line would be anywhere near:
\begin{itemize}
\item End of November: worked out deconvolution part of the program. Have the results in report.
\item Christmas holiday: cleaned up/generalized and completely understand Lauriens program. This will form the basis of the second part of my program.
\item End of February: have first results
\item March/April: difficult for me to oversee, probably working out extra functionals and extending program. Also writing report.
\item Beginning of May: finalizing results and program, completing report.
\item Half May: first draft report
\item End of May: final report and defence.
\end{itemize}

\section{Introduction}
The supervisor on my project is Prof. Sonia Conesa Boj. Furthermore, I will be supervised in an unofficial capacity by dr. Juan Rojo of the Nikhef Institute in Amsterdam. The project is a continuation of the thesis project of Laurien Roest, whose paper can be found at .......



One of the features of the TEM is to obtain electron energy loss spectra of the reviewed sample. This spectrum documents the loss in kinetic energy of electrons travelling through a sample. \cite{egerton_article}. Interpretation of these spectra can help in determining the composition and structure of the evaluated sample. 






\section{Background info}
In the sections below, you will find a first overview of some of the critical background information.

\subsection{TEM} % (fold)
\label{ssub:tem}
The transmission electron microscope (TEM) projects a beam of electron trough an sample, to record the transmitte electrons on the opposite side of the sample to form an image. Due to the fact that the de Broglie wavelenghts of electrons (~$10^{-2}$) are factors smaller than the wavelength of photons ($~10^2$), mugh higher resolutions can be obtained than with light microscopy. The resolution of a TEM is mostly determined by the focussing power of the electron beam. For TEMs with monochromator, such as the TEM available at the Conesa Boj Laberatorium, resolutions of 0.1nm can be obtained, small enough to image single atoms.


% subsubsection tem (end)


\subsubsection{EELS of the TEM}
The electron energy loss spectrum (EELS) documents the loss in kinetic energy of electrons in their path through a specimen. Due to the quantification of energylevels at elementary level, this energy loss manifests in peaks in the spectrum. The broadness of the observed peaks mainly comes from the inherent imperfect electron source, which transmits electrons in a certain energy distribution focussed around the target energy. The analysis of these peaks gives inside in what energy levels are present in the sample, and with that gives inside in the composition and structure of the specimen.




\subsubsection{Low loss spectrum}
Much of the interesting information in the EELS is nested in the low loss part (<50eV) of the spectrum. Here, ones finds info on bandgaps, plasmons and exitons among others. A significant problem in this part of the spectrum, is that the influence of the zero loss peak (ZLP) is non-negliglable. Older tactices to avoid this problem, are centered around fitting the ZLP in each individual spectrum, and subseqently substracting it. The most prominent problem with this approach is that there is no indication of the error in the estimation of the ZLP. Therefore, it can not be said with how significant the found peaks near the tail of the ZLP are. In her master thesis, Laurien Roest develloped a method in which the ZLP is approximated by a neural network, which inherently results in a ZLP with error marges. 

Since the ZLP is an indication of the distribution of the energies of the electrons produced by the electron source, the errors around the ZLP trackle trhough in all the rest of the spectrum. Therefore, in my project I will use the program develloped by Laurien to estimate the ZLP, and use this in the detection and quantification of the peaks in the spectrum and the other variables to be extracted from the spectrum.


%!TEX root = report.tex
\subsection{Dielectric function}

The dielectric function, also called permittivity, is a measure for the polarisability of a material. From the dielectric function, a multitude of other characteristics can be calculated. Since the dielectric function can be extracted from the electron energy loss spectrum through Kramer-Kronig analysis, for an image of spectra, the dielectric function can be calculated for each pixel. \cite{egerton_book}



From \cite{brockt_lakner_2000}:

-Cold field STEM?


Uses deconvolution and Kramer-Kronig relations to determine optical properties of wide-gap materials, specifically III-nitrate compounds (important for optoelectronics).

These properties are tested by comparing them to experimental measurements with synchrotron ellipsometry and theoretical studies.

The resolution is proven by determining the optical features of single layers in a heterostructure and in individual defects within wurtize GaN.


The low loss region of EELS is interesting, as it provides information on excitations of electron in the outer shell, which determine the optical properties of the material. Within the low loss region of the EELS, one can distinguish valence electron exitations, single electron excitations (interband transmissions?, depend on critical points in the bandstructure) and collective plasma oscillations (i.e. plasmon excitations?). 

An alternative method to determine the optical features of material is through optical measurements. These offer a significantly higher energy resolution: 0.001eV-0.5eV where state of the art STEM reach just 0.5eV. The spatial resolution of optical measurements however, is significantly worse mainly due to the higher de Broglie wavelength of photons in respect to electrons \cite{iets}: order microns, whereas STEM can reach sub nanometer.


"
By the use of subnanometer electron probes the spatial resolution of the measurements depends on the physical localization of the scattering process itself and thus is in the range of nanometers for low energy excitations." ?? 


In the low loss region of the EELS, the recorded energy loss function depends on the interjoint density of states between valence and conducting electrons. Peaks arise at critical areas where nesting occurs: where the Fermi surface of the valence electrons is (near) parralel to the Fermi surface of the conducting electrons. In the dielectric formulation, this energy loss function can be described as the response of the material to the passing electron probe, which is proportian to the imaginary part of the reciprocal of the dielectric function $\varepsilon$. Through deconvolution of the signal and Kramer-Kronig relations the complete dielectric function can subsequently be retrieved, see Section \ref{sect_K_K}.





\subsubsection{Relations optical properties to dieelectric function}
From \cite{brockt_lakner_2000}:
The diwlectric function is related to a variety of optical parameter. For example, the relation to refractive index $n$, and absorbtion index $k$, are given by:

\begin{equation}
\begin{aligned}
	\operatorname{Re}[\varepsilon]&=n^{2}-k^{2},\\
	\operatorname{Im}[\varepsilon]&=2 n k.
\end{aligned}
\end{equation}

From these, other optical parameters, such as reflectivity $R$, can be derived:

\begin{equation}
R=\frac{(n-1)^{2}+k^{2}}{(n+1)^{2}+k^{2}}.
\end{equation}


\section{Calculations on the spectrum} % (fold)
\label{sec:calculations_on_the_spectrum}

\subsection{Approximating continious Fourier transform with dicrete Fourier transform}
Lets start with a small refresher on Fourier transforms, and evaluate how to approximate the continious Fourier transform with the discrete Fourier transform.

\subsubsection{Original definitions}
The continous Fourier transfrom (CFT) of function $f(x)$ is defined as:

\begin{equation}\label{eq_def_CFT}
\mathcal{F}\left\{f(x)\right\} = F(\nu) = \int^{\infty}_{-\infty} \operatorname{e}^{-i2\pi\nu x} f(x) dx.
\end{equation}


The inverse of the CFT is given by:

\begin{equation}\label{eq_def_iCFT}
\mathcal{F}^{-1}\left\{F(\nu)\right\} = f(x) = \int^{\infty}_{-\infty} \operatorname{e}^{i2\pi\nu x} F(\nu) d\nu.
\end{equation}

The discrete Fourtier transform (DFT) of discrete function $f[n]$ defined on $n \in \{0, ..., N-1\}$ is given by:

\begin{equation}\label{eq_def_DFT}
\operatorname{DFT}\left\{f[n]\right\} = F[k] = \sum^{N-1}_{n=0} \operatorname{e}^{-i2\pi kn} f[n], \forall k \in \{0, ..., N-1\}.
\end{equation}


The inverse of the CFT is given by:

\begin{equation}\label{eq_def_iDFT}
\operatorname{DFT}^{-1}\left\{F[k]\right\} = f[n] = \sum^{N-1}_{k=0} \operatorname{e}^{i2\pi kn} F[k], \forall n \in \{0, ..., N-1\}.
\end{equation}

\subsubsection{Discrete approximation CFT}


If one approximates function $f(x)$ by $f[x_0 + n\Delta x], \forall n \in \{0,...,N-1\}$, and $\nu$ by $k\Delta \nu, \forall k \in \{0, ..., N-1\}$, where:

\begin{equation}
\Delta x \Delta \nu = \frac{1}{N},
\end{equation}


starting from eq. \eqref{eq_def_CFT}, one can obtain:
\begin{equation}\label{eq_approx_CFT}
\begin{aligned}
	F(\nu) \approx F[k\Delta \nu] &= \sum_{n=0}^{N-1} \Delta x f[x_0 +n\Delta x] \operatorname{exp} \left[-i2\pi k \Delta \nu (x_0 +n \Delta x)\right],\\
	&= \Delta x \operatorname{exp}\left[-i 2\pi k \Delta \nu x_0\right] \sum_{n=0}^{N-1} \operatorname{exp}\left[-i2\pi nk/N\right] f[x_0 +n\Delta x], \\
	&= \Delta x \operatorname{exp}\left[-i 2\pi k \Delta \nu x_0\right] \operatorname{DFT}\left\{f[n]\right\}.
\end{aligned}
\end{equation}

Similary, we can reobtain the original function $f(x) \approx f[n\Delta x + x_0]$ from the approximation of $F(\nu) \approx F[k\Delta \nu]$, using eq. \eqref{eq_def_iCFT}:

\begin{equation}
	\begin{aligned}
	f(x) \approx f[n\Delta x+ x_0] &= \sum_{k=0}^{N-1} \Delta \nu F[k\Delta \nu] \operatorname{exp}\left[i2\pi k\Delta \nu (n\Delta x + x_0)\right]\\
	&= \Delta\nu \sum_{k=0}^{N-1}F[k\Delta\nu]\operatorname{exp}\left[i2\pi k\Delta \nu x_0\right]\operatorname{exp}\left[i2\pi kn/N\right].
	\end{aligned}
\end{equation}

Defining:
\begin{equation}
G[k\Delta\nu]  = \operatorname{exp}\left[i2\pi k \Delta\nu x_0\right] F[k\Delta\nu],
\end{equation}

we find:

\begin{equation}
	\begin{aligned}
	f(x) \approx f[n\Delta x+ x_0] &=  \Delta\nu \sum_{k=0}^{N-1}G[k\Delta\nu]\operatorname{exp}\left[i2\pi kn/N\right]\\
	&= \Delta\nu \operatorname{DFT}^{-1}\left\{G[k\Delta\nu]\right\}
	\end{aligned}
\end{equation}



\subsubsection{Convolutions}
\paragraph{Definition of concolution}

The concolution of two functions $f(x)$ and $g(x)$ is defined as te integral:

\begin{equation}\label{eq_conv_inf}
h(x) = \int_{-\infty}^\infty f(\bar{x})g(x-\bar{x}) d\bar{x}.
\end{equation}

Assuming $f(x)$ and $g(x)$ have a limited domain, $h(n)$ will also have a limited domain, given by:

\begin{equation}
	\begin{aligned}
		f(x) &= \begin{cases}
			\begin{aligned}
				f(x), \quad &x_{0,f} \leq x \leq x_{1,f}\\
				0, \quad &x < x_{0,f} \vee x>x_{1,f}
			\end{aligned}
		\end{cases},\\
		g(x) &= \begin{cases}
			\begin{aligned}
				g(x), \quad &x_{0,g} \leq x \leq x_{1,g}\\
				0, \quad &x < x_{0,g} \vee x>x_{1,g}
			\end{aligned}
		\end{cases},\\
		h(x) &= \begin{cases}
			\begin{aligned}
				h(x), \quad &x_{0,f} + x_{0,g} \leq x \leq x_{1,f} + x_{1,g}\\
				0, \quad &x < x_{0,f} +x_{0,f} \vee x>x_{1,f} + x_{1,g}
			\end{aligned}.
		\end{cases}
	\end{aligned}
\end{equation}

This simplifies eq. \eqref{eq_conv_inf} to:

\begin{equation}
h(x) = \int_{x_{0,f}}^{x_{1,f}} f(\bar{x})g(x-\bar{x}) d\bar{x}.
\end{equation}







Futhermore, the Fourier transform of the convolution has the beautiful property that, for the limited domain functions, but this can be extended into infinity:

\begin{equation}\label{eq_conv_CFT}
	\begin{aligned}
		H(\nu) &= \int_{x_{0,1} + x_{0,g}}^{x_{1,f} + x_{1,g}} h(x) \operatorname{exp}\left[-i2\pi\right]dx,\\
		&= \int_{x_{0,1} + x_{0,g}}^{x_{1,f} + x_{1,g}} \int_{x_{0,1}}^{x_{1,f}} f(\bar{x})g(x-\bar{x}) d\bar{x} \operatorname{exp}\left[-i2\pi\nu x\right]dx,\\
		&= \int_{x_{0,1}}^{x_{1,f}} \int_{{x_{0,1} + x_{0,g}}}^{x_{1,f} + x_{1,g}} g(x-\bar{x}) \operatorname{exp}\left[-i2\pi\nu x\right] dx \quad f(\bar{x}) d\bar{x} .
	\end{aligned}
\end{equation}

Defining $\hat{x} = x-\bar{x}$, evaluation the limits on the integrals and realising $d\hat(x) = dx$, you obtain:

\begin{equation}
	\begin{aligned}
		H(\nu) &=\int_{x_{0,f}}^{x_{1,f}} \int_{x_{0,g}}^{x_{1,g}} g(\hat{x}) \operatorname{exp}\left[-i2\pi\nu(\hat{x} + \bar{x})\right] d\hat{x} \quad f(\bar{x}) d\bar{x}, \\
		&=  \int_{x_{0,g}}^{x_{1,g}} g(\hat{x}) \operatorname{exp}\left[-i2\pi\nu\hat{x}\right]  d\hat{x} \int_{x_{0,f}}^{x_{1,f}} f(\bar{x}) \operatorname{exp}\left[-i2\pi\nu\bar{x}\right] d\bar{x}, \\
		&= F(\nu) G(\nu).
	\end{aligned}
\end{equation}




\paragraph{Discretisation of the convolution and its CFT}
Again, we can approximate $f(x), g(x)$ and $h(n)$ by:

\begin{equation}
\begin{aligned}
	f(x) \approx f[n_f(\Delta x)_f + x_{0,f}], &\forall n_f \in \{0,N_f-1\}, (\Delta x)_f = (x_{1,f}-x_{0,f})/N_f,\\
	g(x) \approx g[n_g(\Delta x)_g + x_{0,g}], &\forall n_g \in \{0,N_g-1\}, (\Delta x)_g = (x_{1,g}-x_{0,g})/N_g.
\end{aligned}
\end{equation}

If we ensure that $(\Delta x)_f = (\Delta x)_g \equiv \Delta x$, we can approximate $h(n)$:

\begin{equation}
	h(x) \approx h[n_h\Delta x + x_{0,f} + x_{0,g}], \forall n_h \in \{0,N_f+N_g-1\},
\end{equation}

where

\begin{equation}
	h[n_h\Delta x + x_{0,f} + x_{0,g}] = \sum_{k=0}^{N_f-1}f[k\Delta x + x_{0,f}]g[n\Delta x + x_{0,g} - k\Delta x ]\Delta x.
\end{equation}


To discretisize the continious fourier transfrom, we once again need to define $\Delta\nu$, but mind the difference for $\Delta\nu$ for the discretisations of $f(x)$ or $g(x)$ and $h(x)$:

\begin{equation}\label{ex_def_deltanu}
\begin{aligned}
(\Delta\nu)_f &= \frac{1}{N_f\Delta x} \\
(\Delta\nu)_g &= \frac{1}{N_g\Delta x} \\
(\Delta\nu)_h &= \frac{1}{(N_f+N_g)\Delta x} \\
&= \left((\Delta\nu)_f^{-1} + (\Delta\nu)_g^{-1}\right)^{-1}
\end{aligned}
\end{equation}


Now the discrete approximation of the CFT of the convolution (given by eq. \eqref{eq_conv_CFT}) is given by:


\begin{equation}
\begin{aligned}
&H(\nu) \approx H[k(\Delta\nu)_h] = \sum_{n=0}^{N_f+N_g-1} h[n\Delta x + x_{0,f} + x_{0,g}]\operatorname{exp}\left[-i2\pi k(\Delta\nu)_h (n\Delta x + x_{0,f} + x_{0,g} \right] \Delta x, \\
&= \sum_{n=0}^{N_f+N_g-1} \sum_{l=0}^{N_f-1}f[l\Delta x + x_{0,f}]g[(n-l)\Delta x + x_{0,g}] \operatorname{exp}\left[-i2\pi k(\Delta\nu)_h (n\Delta x + x_{0,f} + x_{0,g}) \right] (\Delta x)^2.
\end{aligned}
\end{equation}


Defining $m=n-l$, and swapping summation order, you obtain:

\begin{equation}
\begin{aligned}
H[k(\Delta\nu)_h] = (\Delta x)^2&\operatorname{exp}[-2\pi i k(\Delta\nu)_h x_{0,f}] \sum_{l=0}^{N_f-1} f[l(\Delta\nu)_h + x_{0,f}] \\
& \operatorname{exp}[-2\pi i k(\Delta\nu)_h x_{0,g}]  \sum_{m=-l}^{N_f+N_g-l-1}g[m\Delta x + x_{0,g}] \operatorname{exp}[-2\pi i k(\Delta\nu)_h (m+l) \Delta x],
\end{aligned}
\end{equation}


reevaluating the boundries on the second summation, inputing eq. \eqref{ex_def_deltanu} and shuffling above result leads to:
\begin{equation}\label{eq_CFT_approx_conv}
\begin{aligned}
H[k(\Delta\nu)_h] = \operatorname{exp}&[-2\pi i k(\Delta\nu)_h x_{0,f}] \sum_{l=0}^{N_f-1} f[l(\Delta\nu)_h + x_{0,f}]\operatorname{exp}[-2\pi i k l/(N_f+N_g)] \\
& \operatorname{exp}[-2\pi i k(\Delta\nu)_h x_{0,g}]  \sum_{m=0}^{N_g-1}g[m\Delta x + x_{0,g}] \operatorname{exp}[-2\pi i kh m/(N_f+N_g)].
\end{aligned}
\end{equation}

Looking at the above equation, one almost recognizes a multiplication of two approximate CFT's (see eq. \eqref{eq_approx_CFT}). However, there is a significant difference, which means it is not a simple multiplication of the two CFT's: where in the CFT of the singular function, you see that $\Delta\nu = 1/N_f$ or $\Delta\nu = 1/N_g$ resectivily, whereas in eq. \eqref{eq_CFT_approx_conv}, $\Delta\nu = 1/(N_f + N_g)$, as given by eq. \eqref{ex_def_deltanu}. Nor is it a simple factor with which the CFT's could be multiplied: the factor is present in the exponential within the summations. Each single exponential term in the summations need to raised to the power $(N_f+Ng)/N_f$ and $(N_f+Ng)/N_g$ respectivily.% \textcolor{red}{THIS IS THE PROBLEM.}


%\paragraph{Possible solution}
%So what if we add zeros at the beginning/end of both $f[n\Delta x + x_{0,f}]$ and $g[n\Delta x +x_{0,g}$, whilst adjusting $x_{0,f}$ and $x_{0,g}$ accordingly. Since the final convolution has the domain of $[x_{0,f}+x_{0,g},x_{1,f}+x_{1,g}]$, instincivly it'd make sense to adjust 




%Mathematically, I feel like there are a couple of choices one could make regarding the extentions that should not matter in the results

\subsubsection{Convolution of two gaussians}
Defining:

\begin{equation}\label{eq_def_gauss}
f(x)=\frac{A_{f}}{\sqrt{2\pi}\sigma_f} \exp{\left[-\frac{\left(x-\mu_{f}\right)^{2}}{2 \sigma_{f}^{2}}\right]},
\end{equation}

we can find the Fourier transform of $f(x)$ as:

\begin{equation}\label{eq_FT_gauss}
	\mathcal{F}\{f(x)\} \equiv F(\nu) = A_f \exp{\left[-2 \pi i \nu \mu_{f}\right]} \exp{\left[-2 \pi^{2} \sigma_{f}^{2} \nu^{2}\right]},
\end{equation}

which is in itself a gaussian again. We can do the same for a function $g(x)$:

\begin{equation}
g(x)=\frac{A_{g}}{\sqrt{2\pi}\sigma_g} \exp{\left[-\frac{\left(x-\mu_{g}\right)^{2}}{2 \sigma_{g}^{2}}\right]},
\end{equation}

giving:
\begin{equation}
	\mathcal{F}\{g(x)\} \equiv G(\nu) = A_g \exp{\left[-2 \pi i \nu \mu_{f}\right]} \exp{\left[-2 \pi^{2} \sigma_{g}^{2} \nu^{2}\right]}.
\end{equation}


Then the Fourier transform of $f(x)$ and $g(x)$ is given by:
\begin{equation}
\begin{aligned}
\mathcal{F}\{f(x)^*g(x)\} &= F(\nu)G(\nu), \\
 %&= A_f A_g \pi \sigma_f \sigma_g \exp{\left[-2 \pi i \nu \mu_{f}\right]} \exp{\left[-2 \pi^{2} \sigma_{f}^{2} \nu^{2}\right]} \exp{\left[-2 \pi i \nu \mu_{g}\right]} \exp{\left[-2 \pi^{2} \sigma_{g}^{2} \nu^{2}\right]},\\
 &=A_f A_g \exp{\left[-2 \pi i \nu\left(\mu_{f}+\mu_{g}\right)\right]} \exp{\left[-2 \pi^{2}\left(\sigma_{f}^{2}+\sigma_{g}^{2}\right) \nu^{2}\right]}.
\end{aligned}
\end{equation}

In the equation above, a gaussian can be recognised. This means you can write it as:

which you can write as:

\begin{equation}
F(\nu)G(\nu) = \frac{A_C}{\sqrt{2\pi}\sigma_C} \exp{\left[\frac{-(x-\mu_C)^2}{2\sigma_C^2}\right]},
\end{equation}

with:

\begin{equation}
\begin{aligned}
A_C &= A_f A_g \sqrt{2\pi}\sigma_C \exp{\left[-2 \pi i \nu\left(\mu_{f}+\mu_{g}\right)\right]} ,\\
\mu_C &= 0,\\
\sigma_C &= \frac{1}{2\pi(\sigma_f^2+\sigma_g^2)^{1/2}}.
\end{aligned}
\end{equation}




The inverse of this Fourier transform now gives the convolution of the two signals, which you can see is again a gaussian.

\begin{equation}
\begin{aligned}
f(x)^* g(x) &=  \mathcal{F}^{-1}\{F(\nu)G(\nu)\} \\
&= \frac{A_f A_g}{\sqrt{2\pi\left(\sigma_f^{2}+\sigma_g^{2}\right)}} \exp{\left[-\frac{(x-\mu_f-\mu_g)^{2} }{2\left(\sigma_f^{2}+\sigma_g^{2}\right)}\right]},
\end{aligned}
\end{equation}


which you can write as:

\begin{equation}\label{eq_gauss_conv}
\begin{aligned}
f(x)^* g(x) &= \frac{A_c}{\sqrt{2\pi}\sigma_c} \exp{\left[-\frac{(x-\mu_c)^2}{2\sigma_c^2}\right]},\\
A_c &= A_f A_g,\\
\mu_c &= \mu_f + \mu_g,\\
\sigma_c &= (\sigma_f^2 + \sigma_g^2)^{1/2}.
\end{aligned}
\end{equation}

\subsubsection{Other calculations on CFTs}
For the deconvolution of the spectra, we will need the devision and logaritmic values of two CFT's. 
















% section calculations_on_the_spectrum (end)
%!TEX root = MEP_intro_reprt.tex
\subsection{Obtaining the single scattering distribution through deconvolution} \label{sect_deconv}

\subsubsection{Build-up of measured spectrum}
When electrons go through the sample, the intensity of electrons that has no inelastic scattering is given by the zero-loss peak: $I_{ZLP}(E)$. The intensity of the electrons that do scatter, $I_{EEL}(E)$, is than dividable in the single scatter intensity, $I_1(E)$, the double scatter intensity, $I_2(E)$, the triple scatter intensity, $I_3(E)$, etc:

\begin{equation}\label{eq_I}
    I(E) = I_{ZLP}(E) + I_{EEL}(E) = I_{ZLP}(E) + \sum_{n=0}^\infty I_n(E).
\end{equation}


The integrated intensity of each n-scattering spectrum $N_n$  depends on the total integrated intensity $N$, assuming independed scattering events, through the bionomal distribution:

\begin{equation}\label{eq_N_n}
    N_n =  \frac{N}{n!} \left(\frac{t}{\lambda}\right)^n \exp{[-t/\lambda]} .
\end{equation}

Here $t$ is the thickness of the sample, and $\lambda$ is the mean free path of electrons in the sample. 
END DISREGARD

Since we know the zero-loss-peak due to Lau \cite{lau}, the response function of the instrument, $R(E)$, is easily deducted by:

\begin{equation}
    R(E) = I_{ZLP}(E)/N_{ZLP},
\end{equation}

where $N_{ZLP}$ is the integrated intensity of $I_{ZLP}(E)$. 


Now we will show how the total recorded spectrum is build up from single-scattering distribution $S(E)$, and the above defined response function $R(E)$. 


The spectrum recorded due to the single scattering events, $J^1(E)$, is these two distributions convoluted:

\begin{equation} \label{eq_I_1}
    I_{1}(E)=R(E)^{*} S(E) \equiv \int_{-\infty}^{\infty} R\left(E-E^{\prime}\right) S\left(E^{\prime}\right) d E^{\prime}.
\end{equation}


It can be easily seen, that as a double-scattering event, is a series of two single-scattering event, the double-scattering intensity is given by the self convolution of the single-scattering intensity, normalised to match eq. \eqref{eq_N_n}, and once again convoluted with the response function:

\begin{equation}
    I_{2}(E)=R(E)^{*} S(E)^{*} S(E) /\left(2 ! N_{ZLP}\right).
\end{equation}

For higher order scattering spectra, this generalises to:


\begin{equation} \label{eq_def_I_n}
    I_{n}(E)=R(E)\big[^{*} S(E)\big]^{n} /\left(n ! N_{ZLP}^{n-1}\right).
\end{equation}

The complete recorded spectrum, neglecting any surface plasmons, is than given by (analogous to eq. \eqref{eq_I}):
\begin{equation} \label{eq_I_E}
    \begin{aligned}
        I(E) &=I_{ZLP}(E)+I^{1}(E)+I^{2}(E)+I^{3}(E)+\cdots \\
        &=I_{ZLP}(E)^{*}\left[\delta(E)+S(E) / N_{ZLP}+S(E)^{*} S(E) /\left(2 ! N_{ZLP}^{2}\right)\right.\\
        &\left.+S(E)^{*} S(E)^{*} S(E) /\left(3 ! N_{ZLP}^{3}\right)+\cdots\right]\\
        &= I_{ZLP}(E)^{*}\left[\delta(E)+ \sum_{n=1}^\infty \delta(E)\big[^{*} S(E)\big]^{n} /\left(n ! N_{ZLP}^{n}\right) \right].
        \end{aligned}
\end{equation}

Since a convolution in energy domain translates to a multiplication in the frequency domain, it makes sense to take the Fourier transform (FT) of the above equation. Eq. \eqref{eq_I_E} than becomes, using the taylor expansion of the exponential function:

\begin{equation}\label{eq_i_nu_exp}
    \begin{aligned}
i(\nu) &=z(\nu)\left\{1+s(\nu) / N_{ZLP}+[s(\nu)]^{2} /\left(2 ! N_{ZLP}^{2}\right)+[s(\nu)]^{3} /\left(3 ! N_{ZLP}^{3}\right)+\cdots\right\} \\
&=z(\nu)\sum_{n=0}^\infty\frac{s(\nu)^n}{n! N_{ZLP}^n}\\
&=z(\nu) \exp \left[s(\nu) / N_{ZLP}\right],
\end{aligned}
\end{equation}

where $i(\nu)$ is the FT of the intensity spectrum, $z(\nu)$ is the FT of the zero-loss peak, and $s(\nu)$ is the FT of the single-scattering distribution.


The single scattering distribution can than be retrieved by rewriting eq. \eqref{eq_i_nu_exp}, and taking the inverse Fourier transform:

\begin{equation}
    s(\nu) = N_{ZLP} \operatorname{ln}\left(\frac{i(\nu)}{z(\nu)}\right),
\end{equation}

\begin{equation}\label{eq_S_E_FT}
    \begin{aligned}
    S(E) &= \mathcal{F}^{-1}\left\{s(\nu)\right\} \\
    &= \mathcal{F}^{-1}\left\{N_{ZLP}\operatorname{ln}\left[\frac{i(\nu)}{z(\nu)}\right]\right\} \\
    &= \mathcal{F}^{-1}\left\{N_{ZLP}\operatorname{ln}\left[\frac{\mathcal{F}\left\{I(E)\right\}}{\mathcal{F}\left\{I_{ZLP}(E)\right\}}\right]\right\}
    \end{aligned}.
\end{equation}


However, eq. \eqref{eq_S_E_FT} only works for an ``ideal" spectrum. Any noise on the spectrum will blow up, as noise normally preveals itself at higher frequencies, and $i(\nu)$ tends towards zero for high frequencies. Therefor, it is advised to calculate not $S(E)$, but $I_1(E)$, by convoluting once more with $I_{ZLP}(E)$, see eq. \eqref{eq_I_1}. \cite{egerton_book}




\paragraph{Discussion points} What is the most official way to calculate the errors in R(E), from the errors in the ZLP? Just calculate for each ZLP and take the std of those, or can you use the error of the ZLP to calculate it at once? Because it comes back in the $N_{ZLP}$ as well.



\subsubsection{Analytical analysis of possible recorded spectra}
To be able to analyse a deconvolution program, it is usefull to create some toymodels which represent different possible $I(E)$, of which we know what the expected outcome is, so we can validate the program written.


\paragraph{Single scattering distribution as Gaussian}
One of the interesting approximations of a spectrum to review as toy model, is a spectrum in which the zero loss peak is a gaussian such that:

\begin{equation}\label{eq_ZLP_gauss}
I_{ZLP}(E) = \frac{N_{ZLP}}{\sqrt{2\pi}\sigma_{ZLP}} \exp{[-(x-\mu_{ZLP})/(2\sigma_{ZLP}^2)]},
\end{equation}

and the single scattering distribution is a gaussian, given by:

\begin{equation}\label{eq_S_gauss_conv}
S(E) = \frac{A_{S}}{\sqrt{2\pi}\sigma_{S}}  \exp{[-(x-\mu_{S})/(2\sigma_{S}^2)]}.
\end{equation}


By combining eq. \eqref{eq_I}, \eqref{eq_gauss_conv}, and eq. \eqref{eq_def_I_n} you obtain for complete recorded spectrum $I(E)$:

\begin{equation}\label{eq_I_gauss_conv}
\begin{aligned}
I(E) &= \sum_{n=0}^{\infty}  \frac{A_{n}}{\sqrt{2\pi}\sigma_n} \exp{\left[-\frac{(x-\mu_{n})^2}{2\sigma_{n}^2}\right]},\\
A_{n} &= \frac{1}{n! N_{ZLP}^n}N_{ZLP} A_S^n, \\
&= \frac{A_S^n}{n!N_{ZLP}^{n-1}}, \\
\mu_{n} &= \mu_{ZLP} + n \mu_S,\\
\sigma_{n} &= (\sigma_{ZLP}^2 + n \sigma_S^2)^{1/2}.
\end{aligned}
\end{equation}




%\begin{equation}\label{eq_I_gauss_conv}
%\begin{aligned}
%I(E) &= \sum_{n=0}^{\infty}  A_{n} \exp{[-(x-\mu_{n})/(2\sigma_{n}^2)]},\\
%A_{n} &= 
%\begin{cases}
%\begin{aligned}
%N_{ZLP}, &n=0, \\ 
%\frac{N_{ZLP}}{n! N_{ZLP}^n} \frac{\left(\sqrt{2 \pi} A_S\right)^n}{\left(\frac{1}{\sigma_{ZLP}^{2}}+\frac{n}{\sigma_S^{2}}\right)^{1/2}}, &n>0, \\
%\end{aligned}
%\end{cases}\\
%\mu_{n} &= \mu_{ZLP} + n \mu_S,\\
%\sigma_{n} &= (\sigma_{ZLP}^2 + n \sigma_S^2)^{1/2}.
%\end{aligned}
%\end{equation}

This means that for an $I(E)$ constructed as the equation above, with our program, we hope to retrieve $S(E)$ as given by \eqref{eq_S_gauss_conv}.

















\paragraph{Recorded inelastic scattering spectrum as Gaussian}
Starting the other way around, with again an $I_{ZLP}(E)$ as given as eq. \eqref{eq_ZLP_gauss}, but now $I_{EEL}(E)$ is also given itself a gaussian, instead of a summation over convolutions of gaussians. Now, we need to follow the route given in the section above to obtain the single scattering distribution $S(E)$:

\begin{equation}
\begin{aligned}
I(E) &= I_{ZLP}(E) + I_{EEL}(E),\\
&= \frac{N_{ZLP}}{\sqrt{2\pi}\sigma_{ZLP}} \exp{[-(x-\mu_{ZLP})/(2\sigma_{ZLP}^2)]} +\frac{A_{EEL}}{\sqrt{2\pi}\sigma_{EEL}} \exp{[-(x-\mu_{EEL})^2/(2\sigma_{EEL}^2)]}.
\end{aligned}
\end{equation}

Following eq. \eqref{eq_S_E_FT}, we need the Fourier transforms of both the $I_{ZLP}(E)$, and $I(E)$, for which we can use eq. \eqref{eq_FT_gauss}:

\begin{equation}\label{eq_i_nu_gauss}
    \mathcal{F}\{I_{ZLP}(E)\} \equiv z(\nu) = N_{ZLP}\exp{\left[-2 \pi i \nu \mu_{ZLP}\right]} \exp{\left[-2 \pi^{2} \sigma_{ZLP}^{2} \nu^{2}\right]},
\end{equation}

\begin{equation}\label{eq_i_EEL_gauss}
    \mathcal{F}\{I_{EEL}\} \equiv i_{EEL}(\nu) = A_{EEL} \exp{\left[-2 \pi i \nu \mu_{EEL}\right]} \exp{\left[-2 \pi^{2} \sigma_{EEL}^{2} \nu^{2}\right]},
\end{equation}

\begin{equation}
\begin{aligned}
\mathcal{F}\{I(E)\} \equiv i(\nu) &= z(\nu) + i_{EEL}(\nu) \\
&= N_{ZLP}  \exp{\left[-2 \pi i \nu \mu_{ZLP}\right]} \exp{\left[-2 \pi^{2} \sigma_{ZLP}^{2} \nu^{2}\right]} + \\
& \quad \quad A_{EEL}  \exp{\left[-2 \pi i \nu \mu_{EEL}\right]} \exp{\left[-2 \pi^{2} \sigma_{EEL}^{2} \nu^{2}\right]}.
\end{aligned}
\end{equation}



Now eq. \eqref{eq_S_E_FT} becomes:

\begin{equation}\label{eq_S_gauss}
\begin{aligned}
S(E) &= \mathcal{F}^{-1}\left\{N_{ZLP}\operatorname{log}\left[\frac{i(\nu)}{z(\nu)}\right]\right\} \\
&= \mathcal{F}^{-1}\left\{N_{ZLP}\operatorname{log}\left[1+ i_{EEL}/z_{\nu}\right]\right\}.
\end{aligned}
\end{equation}

Since both $i_{EEL}(\nu)$ and $z(\nu)$ are gaussians with mean $\mu = 0$, the devision of the two is also an gaussian given by:

\begin{equation}
\begin{aligned}
i_{EEL}/z_{\nu} &= \frac{A_{EEL} \exp{\left[-2 \pi i \nu \mu_{EEL}\right]} \exp{\left[-2 \pi^{2} \sigma_{EEL}^{2} \nu^{2}\right]}}{N_{ZLP} \exp{\left[-2 \pi i \nu \mu_{ZLP}\right]} \exp{\left[-2 \pi^{2} \sigma_{ZLP}^{2} \nu^{2}\right]}} ,\\
&= \frac{A_{EEL}}{N_{ZLP}} \exp{\left[-2 \pi i \nu (\mu_{EEL} - \mu_{ZLP})\right]} \exp{\left[-2 \pi^{2} (\sigma_{EEL}^{2} - \sigma_{ZLP}^{2}) \nu^{2}\right]},
\end{aligned}
\end{equation}

which can be written as, analogous to eq. \eqref{eq_FT_gauss}:



\begin{equation}\label{eq_gauss_conv2}
\begin{aligned}
i_{EEL}(\nu)/z(\nu) &= A_d \exp{[-2\pi i \nu \mu_d]} \exp{\left[ -2\pi^2 \sigma_d^2 \nu^2 \right]},\\
A_d &= \frac{A_{EEL}}{N_{ZLP}},\\
\mu_d &= \mu_{EEL} - \mu_{ZLP},\\
%&=\frac{A_{EEL} }{N_{ZLP}\sqrt{2\pi(\sigma_{EEL}^2 - \sigma_{ZLP}^2)}}\exp{\left[-2 \pi i \nu (\mu_{EEL} - \mu_{ZLP})\right]},\\
\sigma_d &= \sqrt{\sigma_{EEL}^2 - \sigma_{ZLP}^2}.
\end{aligned}
\end{equation}

%\begin{equation}\label{eq_gauss_conv2}
%\begin{aligned}
%i_{EEL}(\nu)/z(\nu) &= A_d \exp{\left[\frac{-x^2}{(2\sigma_d^2)}\right]},\\
%A_d &= \frac{A_{EEL}}{N_{ZLP}}\exp{\left[-2 \pi i \nu (\mu_{EEL} - \mu_{ZLP})\right]},\\
%%&=\frac{A_{EEL} }{N_{ZLP}\sqrt{2\pi(\sigma_{EEL}^2 - \sigma_{ZLP}^2)}}\exp{\left[-2 \pi i \nu (\mu_{EEL} - \mu_{ZLP})\right]},\\
%\sigma_d &= \frac{1}{2\pi(\sigma_{EEL}^2 - \sigma_{ZLP}^2)^{1/2}}.
%\end{aligned}
%\end{equation}





Now realising that $z(\nu)$ is always exponentially bigger than $i_{EEL}(\nu)$, we can use a Taylor expansion to approximate $Log(z+1)$ (with Log(z) the primitive value of the complex logaritm):

\begin{equation}
Log(z+1) = \sum_{n=1}^\infty \frac{(-1)^{n+1}}{n} z^n.
\end{equation}


With the equations above, we can rewrite eq. \eqref{eq_S_gauss}:

\begin{equation}\label{eq_S_gauss_verder}
\begin{aligned}
S(E) &= \mathcal{F}^{-1}\left\{N_{ZLP}\operatorname{log}\left[1+ \frac{i_{EEL}}{z(\nu)}\right]\right\},\\
&= \mathcal{F}^{-1}\left\{ N_{ZLP} \sum_{n=1}^\infty \frac{(-1)^{n+1}}{n} \left(A_d \exp{[-2\pi i \nu \mu_d]} \exp{\left[ -2\pi^2 \sigma_f^2 \nu^2 \right]}\right)^n  \right\},\\
&= \mathcal{F}^{-1}\left\{ N_{ZLP} \sum_{n=1}^\infty \frac{(-1)^{n+1}}{n} A_d^n \exp{[-2\pi i \nu n \mu_d]} \exp{\left[ -2\pi^2 n \sigma_f^2 \nu^2 \right]}^n  \right\},\\
&= \mathcal{F}^{-1}\left\{ F_1(\nu) \right\} + \mathcal{F}^{-1}\left\{ F_2(\nu) \right\} + \mathcal{F}^{-1}\left\{ F_3(\nu) \right\} + ... , \\
&= \sum_1^{\infty} \mathcal{F}^{-1}\left\{ F_n(\nu) \right\} ,
\end{aligned}
\end{equation}



%\begin{equation}\label{eq_S_gauss_verder}
%\begin{aligned}
%S(E) &= \mathcal{F}^{-1}\left\{N_{ZLP}\operatorname{log}\left[1+ \frac{i_{EEL}}{z(\nu)}\right]\right\},\\
%&= \mathcal{F}^{-1}\left\{ N_{ZLP} \sum_{n=1}^\infty \frac{(-1)^{n+1}}{n} \left(A_d \exp{\left[\frac{-x^2}{2\sigma_d^2}\right]}\right)^n  \right\},\\
%&= \mathcal{F}^{-1}\left\{ N_{ZLP}A_d \exp{\left[\frac{-x^2}{2\sigma_d^2}\right]} \right\} - \mathcal{F}^{-1}\left\{N_{ZLP}  A_d^2  \exp{\left[\frac{-2x^2}{2\sigma_d^2}\%right]} \right\} \\
%& \quad \quad \quad \quad+ \mathcal{F}^{-1}\left\{N_{ZLP} A_d^3 \exp{\left[\frac{-3x^2}{2\sigma_d^2}\right]} \right\} +... ,\\
%&= \mathcal{F}^{-1}\left\{ F_1(\nu) \right\} + \mathcal{F}^{-1}\left\{ F_2(\nu) \right\} + \mathcal{F}^{-1}\left\{ F_3(\nu) \right\} + ... , \\
%&= \sum_1^{\infty} \mathcal{F}^{-1}\left\{ F_n(\nu) \right\} ,
%\end{aligned}
%\end{equation}

Where $F_n(\nu)$ can be written as, analogous to eq. \eqref{eq_FT_gauss}:

\begin{equation}
    \begin{aligned}
F_n (\nu)&= A_{n}\exp{[-2\pi i \nu \mu_n]} \exp{[-2\pi^2 \nu^2 \sigma_{n}^2]},\\
A_{n} &= \frac{(-1)^{n+1}}{n} N_{ZLP}A_d^n,\\
&= (-1)^{n+1} \frac{A_{EEL}^n}{n N_{ZLP}^{n-1}}, \\
\mu_n &= n\mu_d,\\
&= n(\mu_{EEL} - \mu_{ZLP}), \\
\sigma_{n} &= \sqrt{n}\sigma_d, \\
&= \sqrt{n(\sigma_{EEL}^2 - \sigma_{ZLP}^2)}.
    \end{aligned}
\end{equation}


%\begin{equation}
%    \begin{aligned}
%F_n (\nu)&= A_{F_n}\exp{[-2\pi i \nu \mu_n]} \exp{[-2\pi^2 i \nu^2 \sigma_{F_n}^2]},\\
%A_{F_n} &= (-1)^n N_{ZLP}A_d^n,\\
%&= N_{ZLP} \frac{A_{EEL}^n}{N_{ZLP}^n}\exp{\left[-2 \pi i \nu n(\mu_{EEL} - \mu_{ZLP})\right]}, \\
%\sigma_{F_n} &= \frac{\sigma_d}{\sqrt{n}}, \\
%&= \frac{1}{2\pi\sqrt{n(\sigma_{EEL}^2 - \sigma_{ZLP}^2)}}.
%    \end{aligned}
%\end{equation}

Using this, and the inverse relation between eq. \eqref{eq_def_gauss} and eq. \eqref{eq_FT_gauss}, one can find the inverses of the Gaussians that build $S(E)$:

All these inverse Fourier transform of gaussians return gaussians on their part, resulting in a single scattering distribution which is a summation of gaussians, with alterating posive and negative amplitude. These gaussians are given by:

\begin{equation}
\mathcal{F}^{-1}\left\{ F_n(\nu) \right\} = \frac{A_n}{\sqrt{2\pi}\sigma_n} \exp{\left[-\frac{(x-\mu_n)^2}{2\sigma_n^2}\right]}.
\end{equation}


Combining all the above, we find single scattering distribution given by:


\begin{equation}
S(E) = \sum_{n=1}^\infty \frac{(-1)^{n+1} A_{EEL}^n}{n N_{ZLP}^{n-1}\sqrt{2\pi n(\sigma_{EEL}^2 - \sigma_{ZLP}^2)}} \exp{\left[-\frac{(x-n(\mu_{EEL} - \mu_{ZLP}))^2}{2n(\sigma_{EEL}^2 - \sigma_{ZLP}^2)}\right]}
\end{equation}







%!TEX root = report.tex
\newpage

\subsection{Spectrum analysis}\label{sect_K_K}
If one ignores the instrumental broadening, surface-mode scattering and the retardation effects, the single scattering spectrum is approached by the single scattering distribution, which in place can be obtained from the recorded energy loss spectrum by the Fourier log method. \cite{egerton_book}

\begin{equation}\label{eq_S_E}
\begin{aligned}
I_{1}(E) & \approx S(E)=\frac{2 N_{ZLP} t}{\pi a_{0} m_{0} v^{2}} \operatorname{Im}\left[\frac{-1}{\varepsilon(E)}\right] \int_{0}^{\beta} \frac{\theta d \theta}{\theta^{2}+\theta_{E}^{2}} \\
\\
&=\frac{N_{ZLP} t}{\pi a_{0} m_{0} v^{2}} \operatorname{Im}\left[\frac{-1}{\varepsilon(E)}\right] \ln \left[1+\left(\frac{\beta}{\theta_{E}}\right)^{2}\right]
\end{aligned}
\end{equation}

In this equation is $J^1(E)$ the single scattering distribution, $S(E)$ the single scattering spectrum, $N_{ZLP}$ the zero-loss intensity, $t$ the sample thickness, $v$ the velocity of the incoming electron, $\beta$ the collection semi angle, $\alpha$ the angular divergence of the incoming beam, and $\theta_E$ the characteristic scattering angle for energy loss $E$. In this equation $\alpha$ is assumed small in comparison with $\beta$. If this is not the case, additional angular corrections are needed. Furthermore, $\theta_E$ is given by:

\begin{equation} \label{eq_th_E}
    \theta_E = E/(\gamma m_0v^2) .
\end{equation}


Furthermore, it should be noted that to retrieve $\operatorname{Re}\left[1/\varepsilon(E)\right]$ from $\operatorname{Im}\left[-1/\varepsilon(E)\right]$, the Kramer-Kronig relations (see Appendix \ref{sect_KK_rel}, eq. \eqref{eq_ch1_1}) should be rewritten to \cite{Dapor2017}:

\begin{equation}\label{eq_kkr_eps}
    \operatorname{Re}\left[\frac{1}{\varepsilon(E)}\right]=1-\frac{2}{\pi} \mathcal{P} \int_{0}^{\infty} \operatorname{Im}\left[\frac{-1}{\varepsilon\left(E^{\prime}\right)}\right] \frac{E^{\prime} d E^{\prime}}{E^{\prime 2}-E^{2}}.
\end{equation}




\subsection{Step 1: rescaling intensity}
The first step of the K-K analysis is now to rewrite Eq. \eqref{eq_S_E} to:

\begin{equation}\label{eq_J_ac}
    I_{1,ac}(E) = \frac{I_1(E)}{\ln \left[1+\left(\frac{\beta}{\theta_{E}}\right)^{2}\right]} =\frac{N_{ZLP} t}{\pi a_{0} m_{0} v^{2}}  \operatorname{Im}\left[\frac{-1}{\varepsilon(E)}\right] .
\end{equation}


As $\theta_E$ scales linearly with $E$, see eq. \eqref{eq_th_E}, the intensity in on the left side of the equation above now relatively increases for high energy loss with respect to low energy loss.


\paragraph{Discussion points} I assume $\beta$ and $v$ are known, and that we do not take a distribution for $v$? 


\subsection{Step 2: extrapolating}
Since the upcoming integrals all extend to infinity, but the data acquisition is inherently up to a finite energy, the spectra need to be extrapolated. An often used form is $AE^{-r}$, where $r=3$ if you follow the Drude-model, or $r$ can be deducted from experimental data.



\subsection{Step 3: normalisation and retrieving $\operatorname{Im}\left[\frac{1}{\varepsilon(E)}\right]$}

Taking $E' = 0$ in \eqref{eq_kkr_eps}, one obtains:

\begin{equation}
    1-\operatorname{Re}\left[\frac{1}{\varepsilon(0)}\right]=\frac{2}{\pi} \int_{0}^{\infty} \operatorname{Im}\left[\frac{-1}{\varepsilon(E)}\right] \frac{d E}{E}.
\end{equation}

Now dividing both sides of Eq. \eqref{eq_J_ac} by the energy, and subsequently integrating them over energy results in a comparable integral:

\begin{equation}\label{eq_J_ac}
    \int_{0}^{\infty} I_{1,ac}(E) \frac{d E}{E}=  \frac{N_{ZLP} t}{\pi a_{0} m_{0} v^{2}}  \int_{0}^{\infty} \operatorname{Im}\left[\frac{-1}{\varepsilon(E)}\right]   \frac{d E}{E} .
\end{equation}

Combining the two leads to:

\begin{equation}
    \frac{\int_{0}^{\infty} I_{1,ac}(E) \frac{d E}{E}}{\frac{\pi}{2}(1-\operatorname{Re}\left[\frac{1}{\varepsilon(0)}\right])} = \frac{N_{ZLP} t}{\pi a_{0} m_{0} v^{2}} \equiv K ,
\end{equation}
in which $K$ is the proportionality constant, used to estimate the absolute thickness if the zero-loss integral and the indicent energy are known. This formula requires $\operatorname{Re}\left[\frac{1}{\varepsilon(0)}\right]$ to be known, as is the case in for example metals ($\operatorname{Re}\left[\frac{1}{\varepsilon_{metal}(0)}\right]\approx 0$). If this is not the case, other options to estimate $K$ will be discussed later on.

This value of $K$, which is constant over $E$, can than in turn be used to retrieve the function of $\operatorname{Im}\left[-\frac{1}{\varepsilon(E)}\right]$ from the observed single scattering energy distribution $J^1(E)$ with eq. \eqref{eq_J_ac}.


\subsection{Step 4: retrieving $\operatorname{Re}\left[\frac{1}{\varepsilon(E)}\right]$ }
Having retrieved $\operatorname{Im}\left[-\frac{1}{\varepsilon(E)}\right]$ from the steps above, one can now use eq. \eqref{eq_kkr_eps} to obtain $\operatorname{Re}\left[\frac{1}{\varepsilon(E)}\right]$, where one must pay attention to avoid including $E=E'$ in the discrete integral over the spectrum, as this is a singularity. To avoid this singularity in a discrete signal, a couple of approaches are possible:
\begin{itemize}
    \item In the integral (for discrete signals: summation) in eq. \eqref{eq_kkr_eps}, simply exclude the $E = E'$ values.
    \item Shift the values of $\operatorname{Re}\left[\frac{1}{\varepsilon(E)}\right]$ to values at $E''_i = (E_i + E_{i+1})$, to make sure to avoid $E'' = E'$ in the summation.
    \item  The dielectric function in the energy domain relate to the dielectric response function $1/\varepsilon(t) -\delta(t)$ through:
    \begin{equation}
        \operatorname{Re}\left[\frac{1}{\varepsilon(E)}\right] = \mathcal{C}\left\{\frac{1}{\varepsilon(t)} - \delta(t)\right\} = \mathcal{F}\{p(t)\},
    \end{equation}
    and 
    \begin{equation}
        \operatorname{Im}\left[\frac{-1}{\varepsilon(E)}\right] = \mathcal{S}\left\{\frac{1}{\varepsilon(t)} - \delta(t)\right\} = i\mathcal{F}\{q(t)\},
    \end{equation}
    where $p(t)$ and $q(t)$ are the even and odd parts respectively of the dielectric response function, and $\mathcal{C}$ and $\mathcal{S}$ are the cosine and sine Fourier transforms respectively. Since the dielectric response function is a response function and therefor causal, it is $0$ for $t<0$. This results in:
    \begin{equation}
        p(t) = \operatorname{sgn}[q(t)].
    \end{equation}
    Combining all this means that one can also obtain $\operatorname{Re}\left[\frac{1}{\varepsilon(E)}\right]$ from $\operatorname{Im}\left[-\frac{1}{\varepsilon(E)}\right]$ by:
    
    \begin{equation}
        \operatorname{Re}\left[\frac{1}{\varepsilon(E)}\right] =\mathcal{C}\left\{\operatorname{sgn}\left[\mathcal{S}^{-1}\left\{\operatorname{Im}\left[\frac{-1}{ \varepsilon(E)}\right]\right\}\right]\right\}.
    \end{equation}
\end{itemize}

\subsection{Step 5: retrieving $\varepsilon$}
The dielectric function  can subsequently be obtained from:

\begin{equation}
    \varepsilon(E)=\varepsilon_{1}(E)+i \varepsilon_{2}(E)=\frac{\operatorname{Re}[1 / \varepsilon(E)]+i \operatorname{Im}[-1 / \varepsilon(E)]}{\{\operatorname{Re}[1 / \varepsilon(E)]\}^{2}+\{\operatorname{Im}[-1 / \varepsilon(E)]\}^{2}}.
\end{equation}


\vspace{2cm}
%\bibliographystyle{ieee}
%\bibliography{bib}
\printbibliography
%\bibliography{bib}
%\bibliographystyle{ieeetr}
%\bibliographystyle{iopart-num}

\end{document}
