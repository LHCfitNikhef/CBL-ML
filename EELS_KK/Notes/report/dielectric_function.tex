%!TEX root = report.tex
\subsection{Dielectric function}

The complex dielectric function is a key chartaristic for dertermining optical features of a material. The complex part of the dielectriic function, $\varepsilon_1$, describes the dielectric permitivity, and is a measure for the polarisability of a material, whearas the imaginary part, $\varepsilon_2$, describes the energy dissipation \cite{potapov}.


From the dielectric function, a multitude of other characteristics can be calculated. Since the dielectric function can be extracted from the electron energy loss spectrum through Kramer-Kronig analysis, for an image of spectra, the dielectric function can be calculated for each pixel. \cite{egerton_book}



From \cite{brockt_lakner_2000}:

-Cold field STEM?


Uses deconvolution and Kramer-Kronig relations to determine optical properties of wide-gap materials, specifically III-nitrate compounds (important for optoelectronics).

These properties are tested by comparing them to experimental measurements with synchrotron ellipsometry and theoretical studies.

The resolution is proven by determining the optical features of single layers in a heterostructure and in individual defects within wurtize GaN.


The low loss region of EELS is interesting, as it provides information on excitations of electron in the outer shell, which determine the optical properties of the material. Within the low loss region of the EELS, one can distinguish valence electron exitations, single electron excitations (interband transmissions?, depend on critical points in the bandstructure) and collective plasma oscillations (i.e. plasmon excitations?). 

An alternative method to determine the optical features of material is through optical measurements. These offer a significantly higher energy resolution: 0.001eV-0.5eV where state of the art STEM reach just 0.5eV. The spatial resolution of optical measurements however, is significantly worse mainly due to the higher de Broglie wavelength of photons in respect to electrons \cite{iets}: order microns, whereas STEM can reach sub nanometer.


"
By the use of subnanometer electron probes the spatial resolution of the measurements depends on the physical localization of the scattering process itself and thus is in the range of nanometers for low energy excitations." ?? 


In the low loss region of the EELS, the recorded energy loss function depends on the interjoint density of states between valence and conducting electrons. Peaks arise at critical areas where nesting occurs: where the Fermi surface of the valence electrons is (near) parralel to the Fermi surface of the conducting electrons. In the dielectric formulation, this energy loss function can be described as the response of the material to the passing electron probe, which is proportian to the imaginary part of the reciprocal of the dielectric function $\varepsilon$. Through deconvolution of the signal and Kramer-Kronig relations the complete dielectric function can subsequently be retrieved, see Section \ref{sect_K_K}.

At the onset of the EELS, the rise of the energy loss function reflects the increase in the joint DOS, and from the shape of it, one can deduct whether it is a direct or indirect bandbap. \cite{denk betere ref,check lau}. Some smaller rises in the low loss region can become more noticable in the imaginary part of the dielectric function, so that indentifing critical points may be aided by evaluating the dielectric funtion, in addition to the EELS itself.



\subsubsection{Relations optical properties to dieelectric function}
From \cite{brockt_lakner_2000}:
The diwlectric function is related to a variety of optical parameter. For example, the relation to refractive index $n$, and absorbtion index $k$, are given by:

\begin{equation}
\begin{aligned}
	\operatorname{Re}[\varepsilon]&=n^{2}-k^{2},\\
	\operatorname{Im}[\varepsilon]&=2 n k.
\end{aligned}
\end{equation}

From these, other optical parameters, such as reflectivity $R$, can be derived:

\begin{equation}
R=\frac{(n-1)^{2}+k^{2}}{(n+1)^{2}+k^{2}}.
\end{equation}
