\documentclass[12pt,a4paper]{article}
%\documentclass[11pt]{iopart}

\usepackage[colorlinks=true, linkcolor=black!50!blue, urlcolor=blue, citecolor=blue, anchorcolor=blue]{hyperref}
\usepackage[font=small,labelfont=bf,margin=0mm,labelsep=period,tableposition=top]{caption}
\usepackage[a4paper,top=3cm,bottom=2.5cm,left=2.5cm,right=2.5cm,bindingoffset=0mm]{geometry}

\usepackage{graphicx}
\usepackage{float}
\usepackage{afterpage}
\usepackage{epsfig,cite}
\usepackage{amssymb}
\usepackage{amsmath}
\usepackage{bm}
%\usepackage{dsfont}
\usepackage{multirow}
\usepackage{url}
\usepackage{xcolor}
\usepackage{float}
\usepackage{afterpage}
\usepackage{ulem}

\usepackage{url}
\usepackage{hyperref}

\usepackage{multirow,booktabs,multirow}

%\bibliographystyle{iopart-num}
\bibliographystyle{JHEP}

%%%%%%%%%%%%%%%%%%%%%%%%%%%%%%%%%%%%%%%%%%%%%%%%%%%%%%%%%%%%%

\def\smallfrac#1#2{\hbox{$\frac{#1}{#2}$}}
\newcommand{\be}{\begin{equation}}
\newcommand{\ee}{\end{equation}}
\newcommand{\bea}{\begin{eqnarray}}
\newcommand{\eea}{\end{eqnarray}}
\newcommand{\ei}{\end{itemize}}
\newcommand{\ben}{\begin{enumerate}}
\newcommand{\een}{\end{enumerate}}
\newcommand{\la}{\left\langle}
\newcommand{\ra}{\right\rangle}
\newcommand{\lc}{\left[}
  \newcommand{\tr}{\toprule}
  \newcommand{\mr}{\midrule}
  \newcommand{\br}{\bottomrule}
\newcommand{\rc}{\right]}
\newcommand{\lp}{\left(}
\newcommand{\rp}{\right)}
\newcommand{\as}{\alpha_s}
\newcommand{\aq}{\alpha_s\left( Q^2 \right)}
\newcommand{\amz}{\alpha_s\left( M_Z^2 \right)}
\newcommand{\aqq}{\alpha_s \left( Q^2_0 \right)}
\newcommand{\aqz}{\alpha_s \left( Q^2_0 \right)}
\def\toinf#1{\mathrel{\mathop{\sim}\limits_{\scriptscriptstyle
{#1\rightarrow\infty }}}}
\def\tozero#1{\mathrel{\mathop{\sim}\limits_{\scriptscriptstyle
{#1\rightarrow0 }}}}
\def\toone#1{\mathrel{\mathop{\sim}\limits_{\scriptscriptstyle
{#1\rightarrow1 }}}}
\def\frac#1#2{{{#1}\over {#2}}}
\def\gsim{\mathrel{\rlap{\lower4pt\hbox{\hskip1pt$\sim$}}
    \raise1pt\hbox{$>$}}}       
\def\lsim{\mathrel{\rlap{\lower4pt\hbox{\hskip1pt$\sim$}}
    \raise1pt\hbox{$<$}}}       
\newcommand{\mrexp}{\mathrm{exp}}
\newcommand{\dat}{\mathrm{dat}}
\newcommand{\one}{\mathrm{(1)}}
\newcommand{\two}{\mathrm{(2)}}
\newcommand{\art}{\mathrm{art}}
\newcommand{\rep}{\mathrm{rep}}
\newcommand{\net}{\mathrm{net}}
\newcommand{\stopp}{\mathrm{stop}}
\newcommand{\sys}{\mathrm{sys}}
\newcommand{\stat}{\mathrm{stat}}
\newcommand{\diag}{\mathrm{diag}}
\newcommand{\pdf}{\mathrm{pdf}}
\newcommand{\tot}{\mathrm{tot}}
\newcommand{\minn}{\mathrm{min}}
\newcommand{\mut}{\mathrm{mut}}
\newcommand{\partt}{\mathrm{part}}
\newcommand{\dof}{\mathrm{dof}}
\newcommand{\NS}{\mathrm{NS}}
\newcommand{\cov}{\mathrm{cov}}
\newcommand{\gen}{\mathrm{gen}}
\newcommand{\cut}{\mathrm{cut}}
\newcommand{\parr}{\mathrm{par}}
\newcommand{\val}{\mathrm{val}}
\newcommand{\reff}{\mathrm{ref}}
\newcommand{\Mll}{M_{ll}}
\newcommand{\extra}{\mathrm{extra}}
\newcommand{\draft}[1]{}
% Added by MU 
\def \a{\alpha}
\def \b{\beta}
\def \g{\gamma}
\def \z{\zeta}
\def \t{{\bf T}} % vector of theoretical predictions
\def \c{{\bf c}} % vector of coefficients of theoretical predictions
\def \y{{\bf y}} % vector of experimental data
\def \s{{\bf \sigma}} % experimental covariance matrix
% Added by JR
\def\lapprox{\lower .7ex\hbox{$\;\stackrel{\textstyle <}{\sim}\;$}}
\def\gapprox{\lower .7ex\hbox{$\;\stackrel{\textstyle >}{\sim}\;$}}
\def\half{\smallfrac{1}{2}}
\def\GeV{{\rm GeV}}
\def\TeV{{\rm TeV}}
\def\ap{{a'}}
\def\vp{{v'}}
\def\e{\epsilon}
\def\d{{\rm d}}
\def\calN{{\cal N}}
\def\shat{\hat{s}}
\def\barq{\bar{q}}
\def\qq{q \bar q}
\def\uu{u \bar u}
\def\dd{d \bar d}
\def\pp{p \bar p}
\def\xa{x_{1}}
\def\xb{x_{2}}
\def\xaa{x_{1}^{0}}
\def\xbb{x_{2}^{0}}
\def\smx{\stackrel{x\to 0}{\longrightarrow}}
\def\Li{{\rm Li}}
\numberwithin{equation}{section}
\numberwithin{figure}{section}
\numberwithin{table}{section}
\newcommand{\tmop}[1]{\ensuremath{\operatorname{#1}}}
\newcommand{\tmtextit}[1]{{\itshape{#1}}}
\newcommand{\tmtextrm}[1]{{\rmfamily{#1}}}
\newcommand{\tmtexttt}[1]{{\ttfamily{#1}}}
\usepackage{tabularx}
\newcolumntype{C}[1]{>{\centering\arraybackslash}p{#1}}
\begin{document}
\newgeometry{top=1.5cm,bottom=1.5cm,left=2.5cm,right=2.5cm,bindingoffset=0mm}




\begin{center}
  {\Large \bf Notes on KK}
\vspace{1.4cm}

Isabel Postmes, Juan Rojo

\vspace{1.0cm}
 
{\it \small

$^{1}$Kavli Institute of Nanoscience, Delft University of Technology, 2628CJ Delft, The
  Netherlands\\[0.1cm]
$^{2}$Nikhef Theory Group, Science Park 105, 1098 XG Amsterdam, The
  Netherlands \\[0.1cm]$^{3}$Department of Physics and Astronomy, VU,
    1081 HV Amsterdam, The Netherlands

}

\vspace{1.0cm}

{\bf \large Abstract}

\end{center}

Notes on KK


\clearpage
\tableofcontents


\section{Obtaining the single scattering distribution}

\subsection{Build-up of measured spectrum}
When electrons go through the sample, the intensity of electrons that has no inelastic scattering is given by the zero-loss peak: $I_{ZLP}(E)$. The intensity of the electrons that do scatter, $I_{EEL}(E)$, is than dividable in the single scatter intensity, $I_1(E)$, the double scatter intensity, $I_2(E)$, the triple scatter intensity, $I_3(E)$, etc:

\begin{equation}\label{eq_I}
    I(E) = I_{ZLP}(E) + I_{EEL}(E) = I_{ZLP}(E) + \sum_{n=0}^\infty I_n(E).
\end{equation}


MAYBE DISREGARD?
The integrated intensity of each n-scattering spectrum $N_n$ \textcolor{red}{is this a logical choice of letter?} depends on the total integrated intensity $N$, assuming independed scattering events, through the bionomal distribution:

\begin{equation}\label{eq_N_n}
    N_n =  \frac{N}{n!} \left(\frac{t}{\lambda}\right)^n \exp{[-t/\lambda]} .
\end{equation}

Here $t$ is the thickness of the sample, and $\lambda$ is the mean free path of electrons in the sample. 
END DISREGARD

Since we know the zero-loss-peak due to Lau \cite{lau}, the response function of the instrument, $R(E)$, is easily deducted by:

\begin{equation}
    R(E) = I_{ZLP}(E)/N_{ZLP},
\end{equation}

where $N_{ZLP}$ is the integrated intensity of $I_{ZLP}(E)$. 


Now we will show how the total recorded spectrum is build up from single-scattering distribution $S(E)$, and the above defined response function $R(E)$. 


The spectrum recorded due to the single scattering events, $J^1(E)$, is these two distributions convoluted:

\begin{equation}
    I_{1}(E)=R(E)^{*} S(E) \equiv \int_{-\infty}^{\infty} R\left(E-E^{\prime}\right) S\left(E^{\prime}\right) d E^{\prime}.
\end{equation}


It can be easily seen, that as a double-scattering event, is a series of two single-scattering event, the double-scattering intensity is given by the self convolution of the single-scattering intensity, normalised to match eq. \eqref{eq_N_n}, and once again convoluted with the response function:

\begin{equation}
    I_{2}(E)=R(E)^{*} S(E)^{*} S(E) /\left(2 ! N_{ZLP}\right).
\end{equation}

For higher order scattering spectra, this generalises to \textcolor{red}{HOW TO WRITE DOWN A SUCCESSION OF CONVOLUTIONS?}:


\begin{equation}
    I_{n}(E)=R(E)^{*} S(E)\big[^{*} S(E)\big]^{n-1} /\left(n ! N_{ZLP}^{n-1}\right).
\end{equation}

The complete recorded spectrum, neglecting any surface plasmons, is than given by (analogous to eq. \eqref{eq_I}):
\begin{equation} \label{eq_I_E}
    \begin{aligned}
        I(E) &=I_{ZLP}(E)+I^{1}(E)+I^{2}(E)+I^{3}(E)+\cdots \\
        &=I_{ZLP}(E)^{*}\left[\delta(E)+S(E) / N_{ZLP}+S(E)^{*} S(E) /\left(2 ! N_{ZLP}^{2}\right)\right.\\
        &\left.+S(E)^{*} S(E)^{*} S(E) /\left(3 ! N_{ZLP}^{3}\right)+\cdots\right]\\
        &= I_{ZLP}(E)^{*}\left[\delta(E)+ \sum_{n=1}^\infty R(E)^{*} S(E)\big[^{*} S(E)\big]^{n-1} /\left(n ! N_{ZLP}^{n-1}\right) \right].
        \end{aligned}
\end{equation}

Since a convolution in energy domain translates to a multiplication in the frequency domain, it makes sense to take the Fourier transform (FT) of the above equation. Eq. \eqref{eq_I_E} than becomes, using the taylor expansion of the exponential function:

\begin{equation}\label{i_nu_exp}
    \begin{aligned}
i(\nu) &=z(\nu)\left\{1+s(\nu) / N_{ZLP}+[s(\nu)]^{2} /\left(2 ! N_{ZLP}^{2}\right)+[s(\nu)]^{3} /\left(3 ! N_{ZLP}^{3}\right)+\cdots\right\} \\
&=z(\nu)\prod_{n=0}^\infty\frac{s(\nu)^n}{n! N_{ZLP}^n}\\
&=z(\nu) \exp \left[s(\nu) / N_{ZLP}\right],
\end{aligned}
\end{equation}

where $i(\nu)$ is the FT of the intensity spectrum, $z(\nu)$ is the FT of the zero-loss peak, and $s(\nu)$ is the FT of the single-scattering distribution.


\paragraph{Discussion points} What is the most official way to calculate the errors in R(E), from the errors in the ZLP? Just calculate for each ZLP and take the std of those, or can you use the error of the ZLP to calculate it at once? Because it comes back in the $N_{ZLP}$ as well.


\section{Kramer-Kronig relations}
The Kramer-Kronig relations are two functions that relate the imaginary part of an complex function to the real part and vice versa. The relations hold as long as the complex function is analytic in the upper half-plane.
The relations for function $\chi(\omega)=\chi_{1}(\omega)+i \chi_{2}(\omega)$, with $\omega$ a complex variable are given by \cite{wikipedia_2020}:

\begin{equation}
    \chi_{1}(\omega)=\frac{1}{\pi} \mathcal{P} \int_{-\infty}^{\infty} \frac{\chi_{2}\left(\omega^{\prime}\right)}{\omega^{\prime}-\omega} d \omega^{\prime},
\end{equation}

and:

\begin{equation}
    \chi_{2}(\omega)=-\frac{1}{\pi} \mathcal{P} \int_{-\infty}^{\infty} \frac{\chi_{1}\left(\omega^{\prime}\right)}{\omega^{\prime}-\omega} d \omega^{\prime}.
\end{equation}

Here $\mathcal{P}$ denotes the Cauchy principal value of the integral. For causal functions, due to (anti)symmetries arrising from its causality, these can be rewritten to \cite{wikipedia_2020}:

\begin{equation}\label{eq_ch1_1}
    \chi_{1}(\omega)=\frac{2}{\pi} \mathcal{P} \int_{0}^{\infty} \frac{\omega^{\prime} \chi_{2}\left(\omega^{\prime}\right)}{\omega^{\prime 2}-\omega^{2}} d \omega^{\prime},
\end{equation}

and:

\begin{equation}
    \chi_{2}(\omega)=-\frac{2}{\pi} \mathcal{P} \int_{0}^{\infty} \frac{\omega \chi_{1}\left(\omega^{\prime}\right)}{\omega^{\prime 2}-\omega^{2}} d \omega^{\prime}.
\end{equation}


Since the single scattering spectrum of a medium can be related to the imaginary part of the complex permittivity, the Kramer-Kronig relations can be used to retrieve energy dependence of the real permittivity \cite{egerton_2011}. 




\section{Spectrum analysis}
If one ignores the instrumental broadening, surface-mode scattering and the retardation effects, the single scattering spectrum is approached by the single scattering distribution, which in place can be obtained from the recorded energy loss spectrum by the Fourier log method. \cite{egerton_2011}

\begin{equation}\label{eq_S_E}
\begin{aligned}
I_{1}(E) & \approx S(E)=\frac{2 N_{ZLP} t}{\pi a_{0} m_{0} v^{2}} \operatorname{Im}\left[\frac{-1}{\varepsilon(E)}\right] \int_{0}^{\beta} \frac{\theta d \theta}{\theta^{2}+\theta_{E}^{2}} \\
\\
&=\frac{N_{ZLP} t}{\pi a_{0} m_{0} v^{2}} \operatorname{Im}\left[\frac{-1}{\varepsilon(E)}\right] \ln \left[1+\left(\frac{\beta}{\theta_{E}}\right)^{2}\right]
\end{aligned}
\end{equation}

In this equation is $J^1(E)$ the single scattering distribution, $S(E)$ the single scattering spectrum, $N_{ZLP}$ the zero-loss intensity, $t$ the sample thickness, $v$ the velocity of the incoming electron, $\beta$ the collection semi angle, $\alpha$ the angular divergence of the incoming beam, and $\theta_E$ the characteristic scattering angle for energy loss $E$. In this equation $\alpha$ is assumed small in comparison with $\beta$. If this is not the case, additional angular corrections are needed. Furthermore, $\theta_E$ is given by:

\begin{equation} \label{eq_th_E}
    \theta_E = E/(\gamma m_0v^2) .
\end{equation}


Furthermore, it should be noted that to retrieve $\operatorname{Re}\left[1/\varepsilon(E)\right]$ from $\operatorname{Im}\left[-1/\varepsilon(E)\right]$, equation \eqref{eq_ch1_1} should be rewritten to \cite{Dapor2017}:

\begin{equation}\label{eq_kkr_eps}
    \operatorname{Re}\left[\frac{1}{\varepsilon(E)}\right]=1-\frac{2}{\pi} \mathcal{P} \int_{0}^{\infty} \operatorname{Im}\left[\frac{-1}{\varepsilon\left(E^{\prime}\right)}\right] \frac{E^{\prime} d E^{\prime}}{E^{\prime 2}-E^{2}}.
\end{equation}



\paragraph{QUESTIONs on my end} Will we be working with $J(E)$, and is there the need for a Fourier method, or is $J^1(E)$ provided? Are additional angular corrections needed? Where does the 1 come from in eq \eqref{eq_kkr_eps}, is it due to the minus sign in the Im[$-1/\varepsilon$]?

\subsection{Step 1: rescaling intensity}
The first step of the K-K analysis is now to rewrite Eq. \eqref{eq_S_E} to:

\begin{equation}\label{eq_J_ac}
    I_{1,ac}(E) = \frac{I_1(E)}{\ln \left[1+\left(\frac{\beta}{\theta_{E}}\right)^{2}\right]} =\frac{N_{ZLP} t}{\pi a_{0} m_{0} v^{2}}  \operatorname{Im}\left[\frac{-1}{\varepsilon(E)}\right] .
\end{equation}


As $\theta_E$ scales linearly with $E$, see eq. \eqref{eq_th_E}, the intensity in on the left side of the equation above now relatively increases for high energy loss with respect to low energy loss. \textcolor{red}{SOMETHING about aperture correction, is that relevant? }


\paragraph{QUESTIONs on my end} I assume $\beta$ and $v$ are known, and that we do not take a distribution for $v$? 


\subsection{Step 2: extrapolating}
Since the upcoming integrals all extend to infinity, but the data acquisition is inherently up to a finite energy, the spectra need to be extrapolated. An often used form is $AE^{-r}$, where $r=3$ if you follow the Drude-model, or $r$ can be deducted from experimental data.



\subsection{Step 3: normalisation and retrieving $\operatorname{Im}\left[\frac{1}{\varepsilon(E)}\right]$}

Taking $E' = 0$ in \eqref{eq_kkr_eps}, one obtains:

\begin{equation}
    1-\operatorname{Re}\left[\frac{1}{\varepsilon(0)}\right]=\frac{2}{\pi} \int_{0}^{\infty} \operatorname{Im}\left[\frac{-1}{\varepsilon(E)}\right] \frac{d E}{E}.
\end{equation}

Now dividing both sides of Eq. \eqref{eq_J_ac} by the energy, and subsequently integrating them over energy results in a comparable integral:

\begin{equation}\label{eq_J_ac}
    \int_{0}^{\infty} I_{1,ac}(E) \frac{d E}{E}=  \frac{N_{ZLP} t}{\pi a_{0} m_{0} v^{2}}  \int_{0}^{\infty} \operatorname{Im}\left[\frac{-1}{\varepsilon(E)}\right]   \frac{d E}{E} .
\end{equation}

Combining the two leads to:

\begin{equation}
    \frac{\int_{0}^{\infty} I_{1,ac}(E) \frac{d E}{E}}{\frac{\pi}{2}(1-\operatorname{Re}\left[\frac{1}{\varepsilon(0)}\right])} = \frac{N_{ZLP} t}{\pi a_{0} m_{0} v^{2}} \equiv K ,
\end{equation}
in which $K$ is the proportionality constant, used to estimate the absolute thickness if the zero-loss integral and the indicent energy are known. This formula requires $\operatorname{Re}\left[\frac{1}{\varepsilon(0)}\right]$ to be known, as is the case in for example metals ($\operatorname{Re}\left[\frac{1}{\varepsilon_{metal}(0)}\right]\approx 0$). If this is not the case, other options to estimate $K$ will be discussed later on.

This value of $K$, which is constant over $E$, can than in turn be used to retrieve the function of $\operatorname{Im}\left[-\frac{1}{\varepsilon(E)}\right]$ from the observed single scattering energy distribution $J^1(E)$ with eq. \eqref{eq_J_ac}.

\textcolor{red}{need to add other estimations of $K$?}

\subsection{Step 4: retrieving $\operatorname{Re}\left[\frac{1}{\varepsilon(E)}\right]$ }
Having retrieved $\operatorname{Im}\left[-\frac{1}{\varepsilon(E)}\right]$ from the steps above, one can now use eq. \eqref{eq_kkr_eps} to obtain $\operatorname{Re}\left[\frac{1}{\varepsilon(E)}\right]$, where one must pay attention to avoid including $E=E'$ in the discrete integral over the spectrum, as this is a singularity. To avoid this singularity in a discrete signal, a couple of approaches are possible:
\begin{itemize}
    \item In the integral (for discrete signals: summation) in eq. \eqref{eq_kkr_eps}, simply exclude the $E = E'$ values.
    \item Shift the values of $\operatorname{Re}\left[\frac{1}{\varepsilon(E)}\right]$ to values at $E''_i = (E_i + E_{i+1})$, to make sure to avoid $E'' = E'$ in the summation.
    \item  The dielectric function in the energy domain relate to the dielectric response function $1/\varepsilon(t) -\delta(t)$ through:
    \begin{equation}
        \operatorname{Re}\left[\frac{1}{\varepsilon(E)}\right] = \mathcal{C}\left\{\frac{1}{\varepsilon(t)} - \delta(t)\right\} = \mathcal{F}\{p(t)\},
    \end{equation}
    and 
    \begin{equation}
        \operatorname{Im}\left[\frac{-1}{\varepsilon(E)}\right] = \mathcal{S}\left\{\frac{1}{\varepsilon(t)} - \delta(t)\right\} = i\mathcal{F}\{q(t)\},
    \end{equation}
    where $p(t)$ and $q(t)$ are the even and odd parts respectively of the dielectric response function, and $\mathcal{C}$ and $\mathcal{S}$ are the cosine and sine Fourier transforms respectively. Since the dielectric response function is a response function and therefor causal, it is $0$ for $t<0$. This results in:
    \begin{equation}
        p(t) = \operatorname{sgn}[q(t)].
    \end{equation}
    Combining all this means that one can also obtain $\operatorname{Re}\left[\frac{1}{\varepsilon(E)}\right]$ from $\operatorname{Im}\left[-\frac{1}{\varepsilon(E)}\right]$ by:
    
    \begin{equation}
        \operatorname{Re}\left[\frac{1}{\varepsilon(E)}\right] =\mathcal{C}\left\{\operatorname{sgn}\left[\mathcal{S}^{-1}\left\{\operatorname{Im}\left[\frac{-1}{ \varepsilon(E)}\right]\right\}\right]\right\}.
    \end{equation}
\end{itemize}

\subsection{Step 5: retrieving $\varepsilon$}
The dielectric function  can subsequently be obtained from:

\begin{equation}
    \varepsilon(E)=\varepsilon_{1}(E)+i \varepsilon_{2}(E)=\frac{\operatorname{Re}[1 / \varepsilon(E)]+i \operatorname{Im}[-1 / \varepsilon(E)]}{\{\operatorname{Re}[1 / \varepsilon(E)]\}^{2}+\{\operatorname{Im}[-1 / \varepsilon(E)]\}^{2}}.
\end{equation}


\vspace{2cm}

\bibliography{bib}


\end{document}
