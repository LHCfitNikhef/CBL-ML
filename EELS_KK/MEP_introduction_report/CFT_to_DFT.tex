\subsection{Approximating continious Fourier transform with dicrete Fourier transform}
Lets start with a small refresher on Fourier transforms, and evaluate how to approximate the continious Fourier transform with the discrete Fourier transform.

\subsubsection{Original definitions}
The continous Fourier transfrom (CFT) of function $f(x)$ is defined as:

\begin{equation}\label{eq_def_CFT}
\mathcal{F}\left\{f(x)\right\} = F(\nu) = \int^{\infty}_{-\infty} \operatorname{e}^{-i2\pi\nu x} f(x) dx.
\end{equation}


The inverse of the CFT is given by:

\begin{equation}\label{eq_def_iCFT}
\mathcal{F}^{-1}\left\{F(\nu)\right\} = f(x) = \int^{\infty}_{-\infty} \operatorname{e}^{i2\pi\nu x} F(\nu) d\nu.
\end{equation}

The discrete Fourtier transform (DFT) of discrete function $f[n]$ defined on $n \in \{0, ..., N-1\}$ is given by:

\begin{equation}\label{eq_def_DFT}
\operatorname{DFT}\left\{f[n]\right\} = F[k] = \sum^{N-1}_{n=0} \operatorname{e}^{-i2\pi kn} f[n], \forall k \in \{0, ..., N-1\}.
\end{equation}


The inverse of the CFT is given by:

\begin{equation}\label{eq_def_iDFT}
\operatorname{DFT}^{-1}\left\{F[k]\right\} = f[n] = \sum^{N-1}_{k=0} \operatorname{e}^{i2\pi kn} F[k], \forall n \in \{0, ..., N-1\}.
\end{equation}

\subsubsection{Discrete approximation CFT}


If one approximates function $f(x)$ by $f[x_0 + n\Delta x], \forall n \in \{0,...,N-1\}$, and $\nu$ by $k\Delta \nu, \forall k \in \{0, ..., N-1\}$, where:

\begin{equation}
\Delta x \Delta \nu = \frac{1}{N},
\end{equation}


starting from eq. \eqref{eq_def_CFT}, one can obtain:
\begin{equation}\label{eq_approx_CFT}
\begin{aligned}
	F(\nu) \approx F[k\Delta \nu] &= \sum_{n=0}^{N-1} \Delta x f[x_0 +n\Delta x] \operatorname{exp} \left[-i2\pi k \Delta \nu (x_0 +n \Delta x)\right],\\
	&= \Delta x \operatorname{exp}\left[-i 2\pi k \Delta \nu x_0\right] \sum_{n=0}^{N-1} \operatorname{exp}\left[-i2\pi nk/N\right] f[x_0 +n\Delta x], \\
	&= \Delta x \operatorname{exp}\left[-i 2\pi k \Delta \nu x_0\right] \operatorname{DFT}\left\{f[n]\right\}.
\end{aligned}
\end{equation}

Similary, we can reobtain the original function $f(x) \approx f[n\Delta x + x_0]$ from the approximation of $F(\nu) \approx F[k\Delta \nu]$, using eq. \eqref{eq_def_iCFT}:

\begin{equation}
	\begin{aligned}
	f(x) \approx f[n\Delta x+ x_0] &= \sum_{k=0}^{N-1} \Delta \nu F[k\Delta \nu] \operatorname{exp}\left[i2\pi k\Delta \nu (n\Delta x + x_0)\right]\\
	&= \Delta\nu \sum_{k=0}^{N-1}F[k\Delta\nu]\operatorname{exp}\left[i2\pi k\Delta \nu x_0\right]\operatorname{exp}\left[i2\pi kn/N\right].
	\end{aligned}
\end{equation}

Defining:
\begin{equation}
G[k\Delta\nu]  = \operatorname{exp}\left[i2\pi k \Delta\nu x_0\right] F[k\Delta\nu],
\end{equation}

we find:

\begin{equation}
	\begin{aligned}
	f(x) \approx f[n\Delta x+ x_0] &=  \Delta\nu \sum_{k=0}^{N-1}G[k\Delta\nu]\operatorname{exp}\left[i2\pi kn/N\right]\\
	&= \Delta\nu \operatorname{DFT}^{-1}\left\{G[k\Delta\nu]\right\}
	\end{aligned}
\end{equation}



\subsubsection{Convolutions}
\paragraph{Definition of concolution}

The concolution of two functions $f(x)$ and $g(x)$ is defined as te integral:

\begin{equation}\label{eq_conv_inf}
h(x) = \int_{-\infty}^\infty f(\bar{x})g(x-\bar{x}) d\bar{x}.
\end{equation}

Assuming $f(x)$ and $g(x)$ have a limited domain, $h(n)$ will also have a limited domain, given by:

\begin{equation}
	\begin{aligned}
		f(x) &= \begin{cases}
			\begin{aligned}
				f(x), \quad &x_{0,f} \leq x \leq x_{1,f}\\
				0, \quad &x < x_{0,f} \vee x>x_{1,f}
			\end{aligned}
		\end{cases},\\
		g(x) &= \begin{cases}
			\begin{aligned}
				g(x), \quad &x_{0,g} \leq x \leq x_{1,g}\\
				0, \quad &x < x_{0,g} \vee x>x_{1,g}
			\end{aligned}
		\end{cases},\\
		h(x) &= \begin{cases}
			\begin{aligned}
				h(x), \quad &x_{0,f} + x_{0,g} \leq x \leq x_{1,f} + x_{1,g}\\
				0, \quad &x < x_{0,f} +x_{0,f} \vee x>x_{1,f} + x_{1,g}
			\end{aligned}.
		\end{cases}
	\end{aligned}
\end{equation}

This simplifies eq. \eqref{eq_conv_inf} to:

\begin{equation}
h(x) = \int_{x_{0,f}}^{x_{1,f}} f(\bar{x})g(x-\bar{x}) d\bar{x}.
\end{equation}







Futhermore, the Fourier transform of the convolution has the beautiful property that, for the limited domain functions, but this can be extended into infinity:

\begin{equation}\label{eq_conv_CFT}
	\begin{aligned}
		H(\nu) &= \int_{x_{0,1} + x_{0,g}}^{x_{1,f} + x_{1,g}} h(x) \operatorname{exp}\left[-i2\pi\right]dx,\\
		&= \int_{x_{0,1} + x_{0,g}}^{x_{1,f} + x_{1,g}} \int_{x_{0,1}}^{x_{1,f}} f(\bar{x})g(x-\bar{x}) d\bar{x} \operatorname{exp}\left[-i2\pi\nu x\right]dx,\\
		&= \int_{x_{0,1}}^{x_{1,f}} \int_{{x_{0,1} + x_{0,g}}}^{x_{1,f} + x_{1,g}} g(x-\bar{x}) \operatorname{exp}\left[-i2\pi\nu x\right] dx \quad f(\bar{x}) d\bar{x} .
	\end{aligned}
\end{equation}

Defining $\hat{x} = x-\bar{x}$, evaluation the limits on the integrals and realising $d\hat(x) = dx$, you obtain:

\begin{equation}
	\begin{aligned}
		H(\nu) &=\int_{x_{0,f}}^{x_{1,f}} \int_{x_{0,g}}^{x_{1,g}} g(\hat{x}) \operatorname{exp}\left[-i2\pi\nu(\hat{x} + \bar{x})\right] d\hat{x} \quad f(\bar{x}) d\bar{x}, \\
		&=  \int_{x_{0,g}}^{x_{1,g}} g(\hat{x}) \operatorname{exp}\left[-i2\pi\nu\hat{x}\right]  d\hat{x} \int_{x_{0,f}}^{x_{1,f}} f(\bar{x}) \operatorname{exp}\left[-i2\pi\nu\bar{x}\right] d\bar{x}, \\
		&= F(\nu) G(\nu).
	\end{aligned}
\end{equation}




\paragraph{Discretisation of the convolution and its CFT}
Again, we can approximate $f(x), g(x)$ and $h(n)$ by:

\begin{equation}
\begin{aligned}
	f(x) \approx f[n_f(\Delta x)_f + x_{0,f}], &\forall n_f \in \{0,N_f-1\}, (\Delta x)_f = (x_{1,f}-x_{0,f})/N_f,\\
	g(x) \approx g[n_g(\Delta x)_g + x_{0,g}], &\forall n_g \in \{0,N_g-1\}, (\Delta x)_g = (x_{1,g}-x_{0,g})/N_g.
\end{aligned}
\end{equation}

If we ensure that $(\Delta x)_f = (\Delta x)_g \equiv \Delta x$, we can approximate $h(n)$:

\begin{equation}
	h(x) \approx h[n_h\Delta x + x_{0,f} + x_{0,g}], \forall n_h \in \{0,N_f+N_g-1\},
\end{equation}

where

\begin{equation}
	h[n_h\Delta x + x_{0,f} + x_{0,g}] = \sum_{k=0}^{N_f-1}f[k\Delta x + x_{0,f}]g[n\Delta x + x_{0,g} - k\Delta x ]\Delta x.
\end{equation}


To discretisize the continious fourier transfrom, we once again need to define $\Delta\nu$, but mind the difference for $\Delta\nu$ for the discretisations of $f(x)$ or $g(x)$ and $h(x)$:

\begin{equation}\label{ex_def_deltanu}
\begin{aligned}
(\Delta\nu)_f &= \frac{1}{N_f\Delta x} \\
(\Delta\nu)_g &= \frac{1}{N_g\Delta x} \\
(\Delta\nu)_h &= \frac{1}{(N_f+N_g)\Delta x} \\
&= \left((\Delta\nu)_f^{-1} + (\Delta\nu)_g^{-1}\right)^{-1}
\end{aligned}
\end{equation}


Now the discrete approximation of the CFT of the convolution (given by eq. \eqref{eq_conv_CFT}) is given by:


\begin{equation}
\begin{aligned}
&H(\nu) \approx H[k(\Delta\nu)_h] = \sum_{n=0}^{N_f+N_g-1} h[n\Delta x + x_{0,f} + x_{0,g}]\operatorname{exp}\left[-i2\pi k(\Delta\nu)_h (n\Delta x + x_{0,f} + x_{0,g} \right] \Delta x, \\
&= \sum_{n=0}^{N_f+N_g-1} \sum_{l=0}^{N_f-1}f[l\Delta x + x_{0,f}]g[(n-l)\Delta x + x_{0,g}] \operatorname{exp}\left[-i2\pi k(\Delta\nu)_h (n\Delta x + x_{0,f} + x_{0,g}) \right] (\Delta x)^2.
\end{aligned}
\end{equation}


Defining $m=n-l$, and swapping summation order, you obtain:

\begin{equation}
\begin{aligned}
H[k(\Delta\nu)_h] = (\Delta x)^2&\operatorname{exp}[-2\pi i k(\Delta\nu)_h x_{0,f}] \sum_{l=0}^{N_f-1} f[l(\Delta\nu)_h + x_{0,f}] \\
& \operatorname{exp}[-2\pi i k(\Delta\nu)_h x_{0,g}]  \sum_{m=-l}^{N_f+N_g-l-1}g[m\Delta x + x_{0,g}] \operatorname{exp}[-2\pi i k(\Delta\nu)_h (m+l) \Delta x],
\end{aligned}
\end{equation}


reevaluating the boundries on the second summation, inputing eq. \eqref{ex_def_deltanu} and shuffling above result leads to:
\begin{equation}\label{eq_CFT_approx_conv}
\begin{aligned}
H[k(\Delta\nu)_h] = \operatorname{exp}&[-2\pi i k(\Delta\nu)_h x_{0,f}] \sum_{l=0}^{N_f-1} f[l(\Delta\nu)_h + x_{0,f}]\operatorname{exp}[-2\pi i k l/(N_f+N_g)] \\
& \operatorname{exp}[-2\pi i k(\Delta\nu)_h x_{0,g}]  \sum_{m=0}^{N_g-1}g[m\Delta x + x_{0,g}] \operatorname{exp}[-2\pi i kh m/(N_f+N_g)].
\end{aligned}
\end{equation}

Looking at the above equation, one almost recognizes a multiplication of two approximate CFT's (see eq. \eqref{eq_approx_CFT}). However, there is a significant difference, which means it is not a simple multiplication of the two CFT's: where in the CFT of the singular function, you see that $\Delta\nu = 1/N_f$ or $\Delta\nu = 1/N_g$ resectivily, whereas in eq. \eqref{eq_CFT_approx_conv}, $\Delta\nu = 1/(N_f + N_g)$, as given by eq. \eqref{ex_def_deltanu}. Nor is it a simple factor with which the CFT's could be multiplied: the factor is present in the exponential within the summations. Each single exponential term in the summations need to raised to the power $(N_f+Ng)/N_f$ and $(N_f+Ng)/N_g$ respectivily.% \textcolor{red}{THIS IS THE PROBLEM.}


%\paragraph{Possible solution}
%So what if we add zeros at the beginning/end of both $f[n\Delta x + x_{0,f}]$ and $g[n\Delta x +x_{0,g}$, whilst adjusting $x_{0,f}$ and $x_{0,g}$ accordingly. Since the final convolution has the domain of $[x_{0,f}+x_{0,g},x_{1,f}+x_{1,g}]$, instincivly it'd make sense to adjust 




%Mathematically, I feel like there are a couple of choices one could make regarding the extentions that should not matter in the results

\subsubsection{Convolution of two gaussians}
Defining:

\begin{equation}\label{eq_def_gauss}
f(x)=\frac{A_{f}}{\sqrt{2\pi}\sigma_f} \exp{\left[-\frac{\left(x-\mu_{f}\right)^{2}}{2 \sigma_{f}^{2}}\right]},
\end{equation}

we can find the Fourier transform of $f(x)$ as:

\begin{equation}\label{eq_FT_gauss}
	\mathcal{F}\{f(x)\} \equiv F(\nu) = A_f \exp{\left[-2 \pi i \nu \mu_{f}\right]} \exp{\left[-2 \pi^{2} \sigma_{f}^{2} \nu^{2}\right]},
\end{equation}

which is in itself a gaussian again. We can do the same for a function $g(x)$:

\begin{equation}
g(x)=\frac{A_{g}}{\sqrt{2\pi}\sigma_g} \exp{\left[-\frac{\left(x-\mu_{g}\right)^{2}}{2 \sigma_{g}^{2}}\right]},
\end{equation}

giving:
\begin{equation}
	\mathcal{F}\{g(x)\} \equiv G(\nu) = A_g \exp{\left[-2 \pi i \nu \mu_{f}\right]} \exp{\left[-2 \pi^{2} \sigma_{g}^{2} \nu^{2}\right]}.
\end{equation}


Then the Fourier transform of $f(x)$ and $g(x)$ is given by:
\begin{equation}
\begin{aligned}
\mathcal{F}\{f(x)^*g(x)\} &= F(\nu)G(\nu), \\
 %&= A_f A_g \pi \sigma_f \sigma_g \exp{\left[-2 \pi i \nu \mu_{f}\right]} \exp{\left[-2 \pi^{2} \sigma_{f}^{2} \nu^{2}\right]} \exp{\left[-2 \pi i \nu \mu_{g}\right]} \exp{\left[-2 \pi^{2} \sigma_{g}^{2} \nu^{2}\right]},\\
 &=A_f A_g \exp{\left[-2 \pi i \nu\left(\mu_{f}+\mu_{g}\right)\right]} \exp{\left[-2 \pi^{2}\left(\sigma_{f}^{2}+\sigma_{g}^{2}\right) \nu^{2}\right]}.
\end{aligned}
\end{equation}

In the equation above, a gaussian can be recognised. This means you can write it as:

which you can write as:

\begin{equation}
F(\nu)G(\nu) = \frac{A_C}{\sqrt{2\pi}\sigma_C} \exp{\left[\frac{-(x-\mu_C)^2}{2\sigma_C^2}\right]},
\end{equation}

with:

\begin{equation}
\begin{aligned}
A_C &= A_f A_g \sqrt{2\pi}\sigma_C \exp{\left[-2 \pi i \nu\left(\mu_{f}+\mu_{g}\right)\right]} ,\\
\mu_C &= 0,\\
\sigma_C &= \frac{1}{2\pi(\sigma_f^2+\sigma_g^2)^{1/2}}.
\end{aligned}
\end{equation}




The inverse of this Fourier transform now gives the convolution of the two signals, which you can see is again a gaussian.

\begin{equation}
\begin{aligned}
f(x)^* g(x) &=  \mathcal{F}^{-1}\{F(\nu)G(\nu)\} \\
&= \frac{A_f A_g}{\sqrt{2\pi\left(\sigma_f^{2}+\sigma_g^{2}\right)}} \exp{\left[-\frac{(x-\mu_f-\mu_g)^{2} }{2\left(\sigma_f^{2}+\sigma_g^{2}\right)}\right]},
\end{aligned}
\end{equation}


which you can write as:

\begin{equation}\label{eq_gauss_conv}
\begin{aligned}
f(x)^* g(x) &= \frac{A_c}{\sqrt{2\pi}\sigma_c} \exp{\left[-\frac{(x-\mu_c)^2}{2\sigma_c^2}\right]},\\
A_c &= A_f A_g,\\
\mu_c &= \mu_f + \mu_g,\\
\sigma_c &= (\sigma_f^2 + \sigma_g^2)^{1/2}.
\end{aligned}
\end{equation}

\subsubsection{Other calculations on CFTs}
For the deconvolution of the spectra, we will need the devision and logaritmic values of two CFT's. 













