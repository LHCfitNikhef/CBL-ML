%%%%%%%%%%%%%%%%%%%%%%%%%%%%%%%%%%%%%%%%%
\section{Summary and outlook}
%%%%%%%%%%%%%%%%%%%%%%%%%%%%%%%%%%%%%%%%%
\label{sec:summary}

In this work we have presented a novel, model-independent strategy to parametrise and subtract
the ubiquitous zero-loss peak that dominates  the low-loss region
of EEL spectra.
%
Our strategy is based on machine learning techniques and provides a faithful estimate of the
uncertainties associated to both the input data and the procedure itself,
which can  then be propagated to physical predictions  without any  approximations.
%
We have demonstrated how, in the case of vacuum spectra, our approach
is sufficiently flexible to accomodate several input variables corresponding
to different operation conditions of the microscope.
%
Further, we are able  to reliably
extrapolate our predictions, {\it e.g.} for the  expected FWHM of the ZLP,
to other operation conditions.
%
When applied to spectra recorded on specimens, our approach
makes possible to robustly disentangle the ZLP contribution from
those arising from inelastic scatterings.
%
Thanks to this subtraction, one can fully exploit
the valuable physical information contained in the ultra low-loss region of
the spectra.

Here we have applied this ZLP subtraction
strategy to EEL spectra recorded in  WS$_2$ nanoflowers characterised by a
2H/3R polytypic crystalline structure.
%
First of all, measurements taken in a relatively
thick region of the specimen were used to determine
the local value of the bandgap energy $E_{\rm BG}$
and to assess whether this bandgap is direct or indirect.
%
A model fit to the onset of the inelastic intensity distribution obtains
$E_{\rm BG} \simeq 1.6^{+0.3}_{-0.2}\,{\rm eV}$ and exhibits a marked preference for an indirect bandgap.
%
Our findings are consistent with previous studies, both of theoretical
and of experimental nature, concerning the bandgap structure of bulk WS$_2$.

Subsequently, we have applied our method to a  thinner region of the  WS$_2$ nanoflowers,
specifically a region composed by overlapping petals with varying
thicknesses that can be as small as a few monolayers.
%
We have demonstrated how for such specimens one can exploit the ZLP-subtracted results
to characterise the local excitonic transitions that arise in the ultra-low-loss region.
%
By charting the exciton peaks of 2H/3R polytypic WS$_2$ there,
we identify two strong peaks at $\Delta E\simeq 1.5$ and 2 eV
(and a softer one at 1.7 eV) and show how
these features are consistent when comparing
spatially-separated locations in sample B.
%
Further, since our method provides an associated uncertainty estimate,
one can robustly establish the
statistical significance of each of these
ultra-low-loss region features.

The approach presented in this work could be extended
in several directions.
%
First of all, it would be interesting to test its robustness when additional
operation conditions of the microscope are included as input variables,
and to verify to which extent the ZLP parametrisations obtained for an specific microscope
can be generalised to an altogether different TEM.
%
Further, a non-trivial cross-check of our method would be provided by validating
our predictions for other operation conditions of the microscope, such
as the FWHM as a function of the beam energy $E_b$ of the exposure time
$t_{\rm exp}$ reported in Fig.~\ref{fig:extrapolbeam},
with actual measurements.

Concerning the physical interpretation of the low-loss region of EEL
spectra, our method could be applied to study the bandgap properties 
for different types
of nanostructures built upon TMD materials, such as MoS$_2$ nanowalls~\cite{nanowalls}
and 
vertically-oriented nano-sheets~\cite{D0NR00755B} or
WS$_2$/MoS$_2$ arrays, heterostructures, and ternary alloys.
%
In addition to bandgap characterisation, this ZLP-subtraction
strategy should allow the detailed study
of other phenomena relevant for the interpretation of the low-loss
region such as  plasmons, excitons, phonon interactions, and
intra-band transitions.
%
One could also exploit the subtracted EEL spectra to further characterise
local electronic properties by means of the
 evaluation of the complex dielectric function and its associated
uncertainties in terms of the Kramers-Kronig relations.
%
Finally, these phenomenological studies of local electronic properties should be compared
with {\it ab initio} calculations based
on the same underlying crystalline structure as the studied specimens.

Another possible application of the strategy presented in this work would be the automation of
the study of spectral TEM images,
such as those displayed in the right panels of Fig.~\ref{fig:ws2positions},
where each pixel contains an individual EEL spectrum.
%
Here machine learning methods would provide a  useful handle in order
to identify relevant features of the spectra (peaks, edges, shoulders) with minimal
human intervention (no need to process each spectrum individually) and then determine
how these features vary as we move along different regions of the
nanostructure.
%
Such an approach would combine two important families of machine learning algorithms,
those used for regression, in order to quantify the properties of spectral
features such as width and significance, and those for classification, to identify categories
of distinct features across the spectral image.

\subsection*{Acknowledgments}
We are grateful to Emanuele R. Nocera and Jacob J. Ethier for
assistance in installing {\tt EELSfitter} in the Nikhef computing cluster.
%
L.~R. is grateful to Cas, Agneet, and Aar, for support under all 
(rainy) circumstances.


\subsection*{Funding}

S.~E.~v.~H. and S.~C.-B. acknowledge financial support
from the ERC through the Starting Grant ``TESLA”'', grant agreement
no. 805021.
%
L.~M. acknowledges support from the
Netherlands Organizational for Scientific Research (NWO)
through the Nanofront program.
%
The work of J.~R. has been partially supported by NWO.

\subsection*{Declaration of competing interest}

The authors declare that they have no known competing financial interests or personal relationships that could have appeared to influence the work reported in this paper.

\subsection*{Methods}

{\justify
The EEL spectra used for the training of the vacuum ZLP model presented in Sect.~\ref{sec:results_vacuum} were collected in a ARM200F Mono-JEOL microscope equipped with a GIF continuum spectrometer and operated at 60 kV and 200 kV. For these measurements, a slit in the monochromator of 2.8 $\mu$m was used.
%
The TEM and EELS measurements acquired in Specimen A for the results presented in
Sect.~\ref{sec:results_sample} were recorded in a JEOL 2100F microscope with a cold field-emission
gun equipped with aberration corrector operated at 60 kV. A Gatan GIF Quantum was used for
the EELS analyses. The convergence and collection semi-angles were 30.0 mrad and 66.7 mrad respectively.
%
The TEM and EELS measurements acquired for Specimen B in Sect.~\ref{sec:results_sample}
were recorded using a JEM ARM200F monochromated microscope operated at 60 kV and equipped with
a GIF quantum ERS. The convergence and collection semi-angles were 24.6 mrad and 58.4 mrad respectively
in this case, and the aperture of the spectrometer was set to 5 mm.}

