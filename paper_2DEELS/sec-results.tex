\section{Results}
\label{sec:results_vacuum}

We now move to discuss the application of the strategy presented in the previous
section to the parametrisation of ZLP spectra acquired in vacuum.

Some of the plots that we need for this section are

A bunch of fits (by "fit" I mean the full model say with 500 "replicas" at least) for different values of DeltaEI

A heat map plot showing the calculation bandgap across the spectral image, using the approach in Laurien paper

A heat map plot for the thickness, now computed with the deconvoluted spectra just as with the sum of intensities, checking that the results of the two approaches are consistent

A plot of the dielectric function (with uncertainties) for different locations in the spectral image (representative).

A 2D plot demonstrating that the model interpolates in a sensible manner in the "intensity input"

A heat map with the crossing of the x axis for the real (or was it the imaginary) part of the dielectric function across the image

Some explicit test of the stability of the model fitting, for example comparing the results of two fits trained on different random subsets of spectra

A fit that includes the full covariance matrix in the definition of the chi2, rather than only the diagonal component


\subsection{Direct correlation of structural and electrical properties}

Then at some point we can produce correlation plots, for example assessing whenever there is a correlation between thickness and change in the location of the bandgap etc. We should find measures that allow us to identify
in an automated way once we have these kind of correlations.
%
We can define for example a local correlation coefficient between
two distinct featires of the sample

