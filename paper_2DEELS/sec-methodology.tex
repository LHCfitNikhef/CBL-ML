%%%%%%%%%%%%%%%%%%%%%%%%%%%%%%%%%%%%%%%%%%%%%%%%%%%%%%
\section{Methodology}
%%%%%%%%%%%%%%%%%%%%%%%%%%%%%%%%%%%%%%%%%%%%%%%%%%%%%
\label{sec:methodology}

In this section we present our strategy to parametrise and subtract in a model-independent manner
the zero-loss peak that arises in the low-loss region of EEL spectra by means
of machine learning.



\subsection{Clustering}

Here we explain the $k$-means clustering to classify the individual
spectra into groups.
%
Definition of the figure of merit, determination of the number of clusters,
how to treat outliers within clusters.
%
Maybe compare with other methods for clustering? 

\subsection{Neural network training}

Here we explain that we use a NN with two inputs to train the spectra data,
motivate the choice of preprocessing, validation of the training.

We also explain how we evaluate the total errors and the covariance matrix,
which is used to define the figure of merit. Show that once we include the covariance matrix
we obtain $\chi^2/n_{\rm dat}\simeq 1$.

We explain how as opposed to the previous paper now we don't need to
generate replicas, since we have sufficient spectra per cluster to pick one at random.

We need to revisit the choice of $\Delta E_{\rm I}$ and of  $\Delta E_{\rm II}$, in particular
since we don't have vacuum anymore but some material. So we need to use
a different, more general criteria.

Diffferences in the treatment of the EELS data as compared to the previous paper,
for example now we don't need the binning


\subsection{Calculation of dielectic function}

Here we explain how the subtracted ZLP spectra are used to evaluate the dielectric function,
and its uncertainties.
%
Explain how we determine interesting properties of the dielecttic function
which can be represented as a 2D image


\subsection{Automated feature classification}

We explain how we use ML to indentify relevant features in the spectra, say peaks, and how we can
study how their features vary along the sample,
%
We can use multivariate techniques to identify anomalies in the data.





\subsection{Comparison with Laurien's paper}

Check that with the new code we reproduce in a reasonable
manner the results of Laurien's paper
