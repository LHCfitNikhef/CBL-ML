\documentclass[12pt,a4paper]{article}

%\usepackage[colorlinks=true, linkcolor=black!50!blue, urlcolor=blue, citecolor=blue, anchorcolor=blue]{hyperref}
\usepackage[font=small,labelfont=bf,margin=0mm,labelsep=period,tableposition=top]{caption}
\usepackage[a4paper,top=3cm,bottom=3cm,left=2.5cm,right=2.5cm,bindingoffset=0mm]{geometry}

\usepackage{graphicx,ragged2e}
\usepackage{float}
\usepackage{afterpage}
\usepackage{epsfig,cite}
\usepackage{amssymb}
\usepackage{amsmath}
\usepackage{bm}
\usepackage{dsfont}
\usepackage{multirow}
\usepackage{url}
\usepackage{xcolor}
\usepackage{float}
\usepackage{afterpage}
\usepackage{ulem}
\usepackage{url}
\usepackage{multirow,booktabs,multirow}
\bibliographystyle{JHEP}

%%%%%%%%%%%%%%%%%%%%%%%%%%%%%%%%%%%%%%%%%%%%%%%%%%%%%%%%%%%%%

\def\smallfrac#1#2{\hbox{$\frac{#1}{#2}$}}
\newcommand{\be}{\begin{equation}}
\newcommand{\ee}{\end{equation}}
\newcommand{\bea}{\begin{eqnarray}}
\newcommand{\eea}{\end{eqnarray}}
\newcommand{\ei}{\end{itemize}}
\newcommand{\ben}{\begin{enumerate}}
\newcommand{\een}{\end{enumerate}}
\newcommand{\la}{\left\langle}
\newcommand{\ra}{\right\rangle}
\newcommand{\lc}{\left[}
  \newcommand{\tr}{\toprule}
  \newcommand{\mr}{\midrule}
  \newcommand{\br}{\bottomrule}
\newcommand{\rc}{\right]}
\newcommand{\lp}{\left(}
\newcommand{\rp}{\right)}
\newcommand{\as}{\alpha_s}
\newcommand{\aq}{\alpha_s\left( Q^2 \right)}
\newcommand{\amz}{\alpha_s\left( M_Z^2 \right)}
\newcommand{\aqq}{\alpha_s \left( Q^2_0 \right)}
\newcommand{\aqz}{\alpha_s \left( Q^2_0 \right)}
\def\toinf#1{\mathrel{\mathop{\sim}\limits_{\scriptscriptstyle
{#1\rightarrow\infty }}}}
\def\tozero#1{\mathrel{\mathop{\sim}\limits_{\scriptscriptstyle
{#1\rightarrow0 }}}}
\def\toone#1{\mathrel{\mathop{\sim}\limits_{\scriptscriptstyle
{#1\rightarrow1 }}}}
\def\frac#1#2{{{#1}\over {#2}}}
\def\gsim{\mathrel{\rlap{\lower4pt\hbox{\hskip1pt$\sim$}}
    \raise1pt\hbox{$>$}}}       
\def\lsim{\mathrel{\rlap{\lower4pt\hbox{\hskip1pt$\sim$}}
    \raise1pt\hbox{$<$}}}       
\newcommand{\mrexp}{\mathrm{exp}}
\newcommand{\dat}{\mathrm{dat}}
\newcommand{\one}{\mathrm{(1)}}
\newcommand{\two}{\mathrm{(2)}}
\newcommand{\art}{\mathrm{art}}
\newcommand{\rep}{\mathrm{rep}}
\newcommand{\net}{\mathrm{net}}
\newcommand{\stopp}{\mathrm{stop}}
\newcommand{\sys}{\mathrm{sys}}
\newcommand{\stat}{\mathrm{stat}}
\newcommand{\diag}{\mathrm{diag}}
\newcommand{\pdf}{\mathrm{pdf}}
\newcommand{\tot}{\mathrm{tot}}
\newcommand{\minn}{\mathrm{min}}
\newcommand{\mut}{\mathrm{mut}}
\newcommand{\partt}{\mathrm{part}}
\newcommand{\dof}{\mathrm{dof}}
\newcommand{\NS}{\mathrm{NS}}
\newcommand{\cov}{\mathrm{cov}}
\newcommand{\gen}{\mathrm{gen}}
\newcommand{\cut}{\mathrm{cut}}
\newcommand{\parr}{\mathrm{par}}
\newcommand{\val}{\mathrm{val}}
\newcommand{\reff}{\mathrm{ref}}
\newcommand{\Mll}{M_{ll}}
\newcommand{\extra}{\mathrm{extra}}
\newcommand{\draft}[1]{}
% Added by MU 
\def \a{\alpha}
\def \b{\beta}
\def \g{\gamma}
\def \z{\zeta}
\def \t{{\bf T}} % vector of theoretical predictions
\def \c{{\bf c}} % vector of coefficients of theoretical predictions
\def \y{{\bf y}} % vector of experimental data
\def \s{{\bf \sigma}} % experimental covariance matrix
% Added by JR
\def\lapprox{\lower .7ex\hbox{$\;\stackrel{\textstyle <}{\sim}\;$}}
\def\gapprox{\lower .7ex\hbox{$\;\stackrel{\textstyle >}{\sim}\;$}}
\def\half{\smallfrac{1}{2}}
\def\GeV{{\rm GeV}}
\def\TeV{{\rm TeV}}
\def\ap{{a'}}
\def\vp{{v'}}
\def\e{\epsilon}
\def\d{{\rm d}}
\def\calN{{\cal N}}
\def\shat{\hat{s}}
\def\barq{\bar{q}}
\def\qq{q \bar q}
\def\uu{u \bar u}
\def\dd{d \bar d}
\def\pp{p \bar p}
\def\xa{x_{1}}
\def\xb{x_{2}}
\def\xaa{x_{1}^{0}}
\def\xbb{x_{2}^{0}}
\def\smx{\stackrel{x\to 0}{\longrightarrow}}
\def\Li{{\rm Li}}
\numberwithin{equation}{section}
\numberwithin{figure}{section}
\numberwithin{table}{section}
\newcommand{\tmop}[1]{\ensuremath{\operatorname{#1}}}
\newcommand{\tmtextit}[1]{{\itshape{#1}}}
\newcommand{\tmtextrm}[1]{{\rmfamily{#1}}}
\newcommand{\tmtexttt}[1]{{\ttfamily{#1}}}
\usepackage{tabularx}
\newcolumntype{C}[1]{>{\centering\arraybackslash}p{#1}}
\begin{document}
\newgeometry{top=1.5cm,bottom=1.5cm,left=2.5cm,right=2.5cm,bindingoffset=0mm}

%\title[Charting Electron Energy Loss Spectroscopy with machine learning]{Charting the low-loss region in Electron Energy Loss Spectroscopy with machine learning}

%\author{}
%\address{}

%\ead{s.conesaboj@tudelft.nl}
%\vspace{10pt}
%\begin{indented}
%\item[]September 2020
%\end{indented}


\begin{flushright}
Nikhef/2020-022\\
\end{flushright}
\vspace{0.3cm}

\begin{center}
  {\Large \bf Automated data processing and feature identification\\[0.3cm] in
    EELS spectral images with machine learning}
\vspace{1.4cm}


author list

\vspace{1.0cm}
 
{\it \small

$^{1}$Kavli Institute of Nanoscience, Delft University of Technology, 2628CJ Delft, The
  Netherlands\\[0.1cm]
$^{2}$Nikhef Theory Group, Science Park 105, 1098 XG Amsterdam, The
  Netherlands \\[0.1cm]$^{3}$Department of Physics and Astronomy, VU,
    1081 HV Amsterdam, The Netherlands

}

\vspace{1.0cm}

{\bf \large Abstract}

\end{center}

Spectral images in Electron Energy Loss Spectroscopy (EELS) are two-dimensional
sets of spectra where each pixel corresponds a highly localised region of the analysed sample.
%
Here we present a novel approach to automated data processing and feature identification
in EELS spectral images based on machine learning.
%
The constituent spectra are clustered into groups of pixels associated
to sample regions with similar thickness using unsupervised learning,
and then the zero-loss peak (ZLP) is subtracted in a model-independent manner
by means of deep neural networks.
%
The resulting spectral images are  processed to determine 
local electronic properties such as
the band gap and the dielectric function across the sample,
which are then correlated  with the local thickness and other
relevant structural properties.
%
In addition to providing unique information on the direct correlation
between structural and electrical properties, our approach makes possible
the automated identification of interesting features
in the spectra (say a narrow peak)  then determine
how these features  (peak position and width) vary within different regions of the
sample.
%
This strategy, which can be straightforwardly extended
to higher-dimensional datasets,
is implemented into a new release of the open source
code {\tt EELSfitter}.


\vspace{0.4cm}
\noindent{\it Keywords:} {\small Transmission Electron Microscopy,
Electron Energy Loss Spectroscopy, Neural Networks, Machine Learning, Transition
Metal Dichalcogenides, Bandgap, Dielectric Function.}\\

\noindent
{\it $^{*}$corresponding author:} \url{s.conesaboj@tudelft.nl}

\clearpage
\tableofcontents


\input{sec-introduction.tex}
%%%%%%%%%%%%%%%%%%%%%%%%%%%%%%%%%%%%%%%%%%%%%%%%%%%%%%
\section{A neural network determination of the ZLP}
%%%%%%%%%%%%%%%%%%%%%%%%%%%%%%%%%%%%%%%%%%%%%%%%%%%%%
\label{sec:methodology}

In this section we present our strategy to parametrise and subtract in a model-independent manner
the zero-loss peak that arises in the low-loss region of EEL spectra by means
of machine learning.



\section{Results}
\label{sec:results_vacuum}

We now move to discuss the application of the strategy presented in the previous
section to the parametrisation of ZLP spectra acquired in vacuum.

Some of the plots that we need for this section are

A bunch of fits (by "fit" I mean the full model say with 500 "replicas" at least) for different values of DeltaEI

A heat map plot showing the calculation bandgap across the spectral image, using the approach in Laurien paper

A heat map plot for the thickness, now computed with the deconvoluted spectra just as with the sum of intensities, checking that the results of the two approaches are consistent

A plot of the dielectric function (with uncertainties) for different locations in the spectral image (representative).

A 2D plot demonstrating that the model interpolates in a sensible manner in the "intensity input"

A heat map with the crossing of the x axis for the real (or was it the imaginary) part of the dielectric function across the image

Some explicit test of the stability of the model fitting, for example comparing the results of two fits trained on different random subsets of spectra

A fit that includes the full covariance matrix in the definition of the chi2, rather than only the diagonal component


\subsection{Direct correlation of structural and electrical properties}

Then at some point we can produce correlation plots, for example assessing whenever there is a correlation between thickness and change in the location of the bandgap etc. We should find measures that allow us to identify
in an automated way once we have these kind of correlations.
%
We can define for example a local correlation coefficient between
two distinct featires of the sample


%%%%%%%%%%%%%%%%%%%%%%%%%%%%%%%%%%%%%%%%%
\section{Summary and outlook}
%%%%%%%%%%%%%%%%%%%%%%%%%%%%%%%%%%%%%%%%%
\label{sec:summary}

In this work we have presented a novel, model-independent strategy to parametrise and subtract
the ubiquitous zero-loss peak that dominates  the low-loss region
of EEL spectra.


Say somethihg about the connection of our approach with
indirect Dark Matter searches and how we can efficiently
test new materials that can eventually be used
as dark matter detectors.

\subsection*{Acknowledgments}

We are grateful to Emanuele R. Nocera and Jacob J. Ethier for
assistance in installing {\tt EELSfitter} in the Nikhef computing cluster.


\subsection*{Funding}

S.~E.~v.~H. and S.~C.-B. acknowledge financial support
from the ERC through the Starting Grant ``TESLA”'', grant agreement
no. 805021.
%
L.~M. acknowledges support from the
Netherlands Organizational for Scientific Research (NWO)
through the Nanofront program.
%
The work of J.~R. has been partially supported by NWO.

\subsection*{Declaration of competing interest}

The authors declare that they have no known competing financial interests or personal relationships that could have appeared to influence the work reported in this paper.

\subsection*{Methods}

{\justify
The EEL spectra used for the training of the vacuum ZLP model presented in Sect.~\ref{sec:results_vacuum} were collected in a ARM200F Mono-JEOL microscope equipped with a GIF continuum spectrometer and operated at 60 kV and 200 kV. For these measurements, a slit in the monochromator of 2.8 $\mu$m was used.
%
The TEM and EELS measurements acquired in Specimen A for the results presented in
Sect.~\ref{sec:results_sample} were recorded in a JEOL 2100F microscope with a cold field-emission
gun equipped with aberration corrector operated at 60 kV. A Gatan GIF Quantum was used for
the EELS analyses. The convergence and collection semi-angles were 30.0 mrad and 66.7 mrad respectively.
%
The TEM and EELS measurements acquired for Specimen B in Sect.~\ref{sec:results_sample}
were recorded using a JEM ARM200F monochromated microscope operated at 60 kV and equipped with
a GIF quantum ERS. The convergence and collection semi-angles were 24.6 mrad and 58.4 mrad respectively
in this case, and the aperture of the spectrometer was set to 5 mm.}


\bibliography{EELS_ML}
%\input{EELS_ML.bbl}




\end{document}
