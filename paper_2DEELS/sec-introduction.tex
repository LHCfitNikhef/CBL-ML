\section{Introduction}
\label{sec:introduction}

Electron energy-loss spectroscopy (EELS) within the transmission electron microscope (TEM) provides
a wide range of
valuable information on the structural, chemical, and electronic properties of nanoscale materials.
%
Thanks to recent instrumentation breakthroughs
such as electron monochromators~\cite{Terauchi:2005, Freitag:2005} and aberration correctors~\cite{Haider:1998},
modern EELS analyses can study these properties with highly competitive spatial and spectral resolution.
%
A particularly important region of EEL spectra is
the low-loss region, defined by electrons that have lost a few tens of eV,
$\Delta E\lsim 50$ eV,
following their inelastic interactions with the sample.
%
The analysis of this low-loss region makes possible charting the local
electronic properties of nanomaterials~\cite{Geiger:1967}, from the characterisation of
bulk and surface plasmons~\cite{Schaffer:2008}, excitons~\cite{Erni:2005}, 
inter- and intra-band transitions~\cite{Rafferty:1998},
and phonons to the determination of their bandgap and band structure~\cite{Stoger:2008}.

Provided the specimen is electron-transparent, as required for TEM inspection,
the bulk of the incident electron beam will traverse it
either without interacting or restricted to elastic scatterings with the atoms
of the sample's crystalline lattice.
%
In EEL spectra, these electrons are recorded as a narrow,
high intensity peak centered at energy losses
of $\Delta E\simeq 0$, known as the zero-loss peak (ZLP).
%
The energy resolution of EELS analyses is ultimately determined by
the electron beam size of the system, often expressed in terms
of the full width at half maximum (FWHM) of the
ZLP~\cite{Egerton:2009}.
%
In the low-loss region, the contribution from the ZLP
often overwhelms that from the inelastic scatterings arising from
the interactions of the beam electrons with the sample.
%
Therefore, relevant signals of low-loss phenomena such as excitons,
phonons, and intraband transitions risk becoming drowned
in the ZLP tail~\cite{Abajo:2010}.
%
An accurate removal of the ZLP
contribution is thus crucial in order to accurately chart and identify the features
of the low-loss region in EEL spectra.


In monochromated EELS, the properties of the ZLP depend on the electron energy dispersion,
the monochromator alignment, and the sample thickness~\cite{Park:2008, Stoger:2008}.
%
The first two factors arise already in the absence of a specimen (vacuum operation),
while the third is associated
to interactions with the sample such as atomic scatterings,
phonon excitation, and exciton losses.
%
This implies that EEL measurements in vacuum can be used for calibration purposes
but not to subtract the ZLP from spectra taken on specimens, since their shapes will
in general differ.




Several approaches to ZLP subtraction\cite{Rafferty:2000, Stoger:2008, Egerton:1996} 
have been put forward in the literature.
%
These are often based on specific model assumptions about the ZLP properties, in particular
concerning its parametric functional dependence on the electron energy loss $\Delta E$,
from Lorentzian~\cite{Dorneich:1998}
and power laws~\cite{Erni:2005} to more general multiple-parameter functions~\cite{Benthem:2001}.
%
Another approach is based on mirroring the $\Delta E <0$ region of the spectra, assuming
that the $\Delta E>0$ region is fully symmetric~\cite{Lazar:2003}.
%
More recent studies use integrated software applications for background
subtraction~\cite{Egerton:10.1016/S0304-3991(01)00155-3, Held:2020, Granerod:2018, Fung:2020}.
%
These various methods are however affected by three main limitations.
%
Firstly, their reliance on model assumptions such as
the choice of fit function introduces a methodological
bias whose size is difficult to quantify.
%
Secondly, they lack an estimate of the associated uncertainties, which in turn affects
the reliability of any physical interpretations of the low loss region.
%
Thirdly, {\it ad hoc} choices such as those of the fitting ranges introduce a significant degree of
arbitrariness in the procedure.



In this study we bypass these limitations by developing a model-independent strategy
to realise a multidimensional determination of the ZLP
with a faithful uncertainty estimate.
%
Our approach is based on machine learning (ML) techniques
originally developed in high-energy physics to study the
quark and gluon substructure of protons in particle collisions~\cite{Ball:2008by,Ball:2012cx,Ball:2014uwa,Ball:2017nwa}.
%
It is based on the Monte Carlo replica method to construct a probability
distribution in the space of experimental data and artificial
neural networks as unbiased interpolators to parametrise the ZLP.
%
The end result is
a faithful sampling of the probability distribution in the ZLP space 
which can be used to subtract its contribution to EEL spectra while
propagating the associated uncertainties.
%
One can also extrapolate the predictions from this ZLP parametrisation to other TEM
operating conditions beyond those included in the training dataset.



This work is divided into two main parts.
%
In the first one, we construct a ML model of ZLP spectra acquired
in vacuum, which is able to accommodate an arbitrary number of input
variables corresponding to different operation settings of the TEM.
%
We demonstrate how this model successfully describes the
input spectra and we assess its extrapolation capabilities for other operation
conditions.
%
In the second part, we construct a one-dimensional model
of the ZLP as a function of $\Delta E$ from spectra acquired on two different specimens of
tungsten disulfide (WS$_2$) nanoflowers characterised by a 2H/3R mixed polytypism~\cite{SabryaWS2}.
%
The resulting subtracted spectra are used to determine
the value and nature of the WS$_2$ bandgap in these nanostructures
as well as to map the properties of the associated exciton peaks appearing in the ultra-low
loss region.



This paper is organized as follows.
%
First of all, in Sect.~\ref{sec:tmdeels}
we review the main features of EELS and present
the WS$_2$ nanostructures that will be used as proof of concept of our approach.
%
In Sect.~\ref{sec:methodology} we describe the machine learning methodology
adopted to model the ZLP features.
%
Sects.~\ref{sec:results_vacuum} and~\ref{sec:results_sample} contain
the results of the ZLP parametrisation of spectra acquired
in vacuum and in specimens respectively, which in the latter
case allows us to probe the local electronic properties 
of the WS$_2$ nanoflowers.
%
Finally in Sect.~\ref{sec:summary} we summarise
and outline possible future developments.
%
Our results have been obtained with an open-source {\sc Python} code,
dubbed {\tt EELSfitter}, whose installation and usage instructions
are described in Appendix~\ref{sec:installation}.
%
{\color{red} Furthermore, we discuss the possible role played
  by correlated
uncertainties in the training dataset in Appendix~\ref{app:corr}.}
