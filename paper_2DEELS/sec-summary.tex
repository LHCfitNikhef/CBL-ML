%%%%%%%%%%%%%%%%%%%%%%%%%%%%%%%%%%%%%%%%%
\section{Summary and outlook}
%%%%%%%%%%%%%%%%%%%%%%%%%%%%%%%%%%%%%%%%%
\label{sec:summary}

In this work we have presented a novel, model-independent strategy to parametrise and subtract
the ubiquitous zero-loss peak that dominates  the low-loss region
of EEL spectra.

\subsection*{Acknowledgments}

We are grateful to Emanuele R. Nocera and Jacob J. Ethier for
assistance in installing {\tt EELSfitter} in the Nikhef computing cluster.


\subsection*{Funding}

S.~E.~v.~H. and S.~C.-B. acknowledge financial support
from the ERC through the Starting Grant ``TESLA”'', grant agreement
no. 805021.
%
L.~M. acknowledges support from the
Netherlands Organizational for Scientific Research (NWO)
through the Nanofront program.
%
The work of J.~R. has been partially supported by NWO.

\subsection*{Declaration of competing interest}

The authors declare that they have no known competing financial interests or personal relationships that could have appeared to influence the work reported in this paper.

\subsection*{Methods}

{\justify
The EEL spectra used for the training of the vacuum ZLP model presented in Sect.~\ref{sec:results_vacuum} were collected in a ARM200F Mono-JEOL microscope equipped with a GIF continuum spectrometer and operated at 60 kV and 200 kV. For these measurements, a slit in the monochromator of 2.8 $\mu$m was used.
%
The TEM and EELS measurements acquired in Specimen A for the results presented in
Sect.~\ref{sec:results_sample} were recorded in a JEOL 2100F microscope with a cold field-emission
gun equipped with aberration corrector operated at 60 kV. A Gatan GIF Quantum was used for
the EELS analyses. The convergence and collection semi-angles were 30.0 mrad and 66.7 mrad respectively.
%
The TEM and EELS measurements acquired for Specimen B in Sect.~\ref{sec:results_sample}
were recorded using a JEM ARM200F monochromated microscope operated at 60 kV and equipped with
a GIF quantum ERS. The convergence and collection semi-angles were 24.6 mrad and 58.4 mrad respectively
in this case, and the aperture of the spectrometer was set to 5 mm.}

