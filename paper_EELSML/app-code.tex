
\vspace{1cm}
\hrule
\vspace{1cm}

\begin{center}
  {\bf \LARGE Supplementary Material}
 \end{center}

\section{Installation and usage of {\tt EELSfitter}}
\label{sec:installation}

In this appendix we provide some instructions about the installation
and the usage of the {\tt EELSfitter} code developed
in this work.
%
The code is available from its GitHub repository
\begin{center}
\url{https://github.com/LHCfitNikhef/EELSfitter}
\end{center}
and is composed by a number of {\tt Python} scripts.
%
The code requires a working installation of {\tt Python3} and the following
libraries: {\tt NumPy}, {\tt TensorFlow} (v2), {\tt pandas}, {\tt SciPy} and {\tt scikit-learn}.  


\noindent
\paragraph{\tt Load\_data.py}
%
This script reads the spectrum
intensities and create data-frames to be used for training the neural network.
%
It reads out the EEL spectra intensities, automatically selects the energy loss
at which the peak intensity occurs and shifts the dataset such that
the peak intensity is centered at $\Delta E =$0. 
%
Further, for each spectrum it returns the normalized intensity by normalizing
over the total area under the spectrum. 
%
The output is two datasets, {\tt df} and {\tt df\_vacuum} which contain the 
information on the in-sample and in-vacuum recorded spectra respectively. 
%
The user needs to upload the spectral data in .txt format to the 'Data' folder
and make sure that the vacuum and in-sample spectra are added to the appropriate one.
%
For each of the spectra the minimum and maximum value of the recorded energy 
loss need to be set manually in {\tt Eloss\_min} and {\tt Eloss\_max}.

\noindent
\paragraph{\tt Fitter.ipynb}
%
This script is used to run the neural network training on the data that was 
uploaded using {\tt load\_data.py}.
%
It involves a number of pre-processing steps to determine the hyper-parameters $\Delta E_{\rm I}$
and $\Delta E_{\rm II}$ and then
it automatically prepares and cuts the data before it is fed
to the neural network to start the training.
%
It is structured as follows:

\begin{itemize}
  
\item {\it Importing libraries and spectral data} from the {\bf load\_data.py} script.

\item {\it Evaluate  $\Delta E_{\rm I}$ from the intensity derivatives}.
  %
  In order to determine the value for the hyper-parameter
  $\Delta E_{\rm I}$, a dataframe {\tt df\_dx} is created
and it calculates the derivatives of each of the in-sample recorded spectra, 
stored as {\tt df\_dx['derivative y*']}, where {\tt *} is any of the in-sample recorded spectra.
%
The first crossing of any of the derivatives with zero is determined 
and stored as the value of $\Delta E_{\rm I}$. 

\item {\it Evaluate $\Delta E_{\rm II}$ for the pseudo-data.}
  %
It calculates the mean over all vacuum spectra, {\tt df\_mean}, and the ratio of the 
intensity to the experimental uncertainty for each value of
$\Delta E$, {\tt df\_mean['ratio']}. 
%
The value of $\Delta E_{\rm II}$ is then determined as the energy loss at which this ratio
drops below 1 and  is stored together with the value of $\Delta E_{\rm I}$
as the hyper-parameters for training. 
%
However, if one wishes to use other values for these parameters, for instance for 
cross-validating the best value for $\Delta E_{\rm I}$, these can also be adjusted manually.

\item {\it Experimental data processing.}
%
  The next step is to keep only
  the data points with $\Delta E \le \Delta E_{\rm I}$  
and dropping the points with higher energy losses.
%
Experimental central values and uncertainties are calculated by means of equal width 
discretization, for which the number of bins has to be set as {\tt nbins}. 
%
The default value is 32, which means that 32 training inputs are spread equally
over the range [$ \Delta E_{\rm min}, \Delta E_{\rm I}$]. 
%
Note that the logarithm of the intensity is used as training inputs, because this facilitates
the optimization of the neural network ($I_{\rm EEL}$ being a steeply falling function
of $\Delta E$).
%
The code translates this back to the original intensity values after the training.
%
$N_{\rm pd}$
pseudo datapoints are added in the range $[\Delta E_{\rm II}, \Delta E_{\rm max}]$, where
$ \Delta E_{\rm max}$
is the maximum energy loss value of the recorded spectra. 
%
The values for $N_{\rm pd}$ and $\Delta E_{\rm max}$ should be changed manually by 
setting them in {\tt max\_x} and {\tt N\_pseudo}. 
%
The output is a dataframe {\tt df\_full} containing all training data and pseudo data points, 
corresponding to a total of $N_{\rm in}$ (= $n_{\rm bins}$ + $N_{\rm pd}$) training inputs.

\item {\it Initialize the NN model,} where the code
  defines the neural network architecture and prepares the
data inputs to feed them to the neural network for training. 
%
The function {\tt make\_model()} allows to define the number of hidden layers and 
nodes per layer. The default  architecture is 1-10-15-5-1.

\item {\it Initialize data for NN training.}
  %
  Here the code prepares the recorded spectra to be used
as inputs for the neural network. 
%
First, we initiate placeholders for the variables
{\tt x}, {\tt y} and {\tt sigma} which
allow us to create our operations and 
build our computation graph, without needing the data itself. 
%
The dimension of the placeholder is defined by {\tt $[{\bf None}, dim]$} where 'dim'
should be set to the dimension of the corresponding variable. In this case the input is 
one-dimensional, so dim=1. 
%
These placeholders are used to define {\tt predictions}, which is in fact a placeholder that is used 
later to make predictions on inputs {\tt x}. 
%
Also, we define a vector {\tt predict\_x} that is used to make a direct prediction after training
on each of the replicas. It consists of $N_{\rm pred}$ data points in the energy loss range
{\tt [pred\_min, pred\_max]}. 

\item {\it Create the Monte Carlo replicas.}
  %
  The final step to be taken before we can start training is the creation of
  sample of $N_{\rm rep}$ Monte Carlo replicas of the original EEL spectra,
  following the procedure described in Sect.~\ref{sec:MCreplicas}.
%
This is done automatically using the experimental intensities {\tt train\_y} and uncertainties
{\tt train\_sigma} for a total of {\tt Nrep} replicas. The output is an ($N_{\rm in}, N_{\rm rep}$) 
vector containing all the MC replicas. 

\item {\it Train the neural networks.}
  %
  The final part of the script, where the NN training is  carried out,
  is based on the function {\tt function\_train()} that
  implements the strategy presented in Sect.~\ref{sec:training}.
%
The cost function, optimizer and learning rate are defined here, together with a 'saver' used to 
save the network parameters after each optimization. 
%
We start a loop over {\tt Nrep} replicas to initiate a training session on each of the individual replicas
in series. 
%
For each iteration, the $k$-th replica is selected from the sample of $N_{\rm rep}$ replicas.
%
The data is split into 80\% training and 20\% validation data, this partition is done 
at random for each replica. The resulting {\tt train\_y} and {\tt test\_y} arrays are used
as training and validation labels.
%
The total number of training epochs per session is defined in {\tt training\_epochs}.
%
The script displays intermediate 
results after each number of epochs defined by {\tt display\_step}. 
%
Running the session object over the optimizer and cost function requires knowledge about the values of {\tt x} and {\tt sigma}, which 
are defined inside the {\tt feed\_dict} argument. 
%
After each epoch the average training  validation costs are evaluated
and the network parameters  updated accordingly.

Once the maximum number of epochs 
had been reached, the optimal stopping point is determined by 
taking the absolute minimum of the validation cost
and restoring the corresponding network parameters by means of the 'saver' function.
%
From this network graph, one can directly output the prediction on the values of {\tt train\_x} and
the results are stored in the array {\tt predictions\_values}.
%
It is also possible to make predictions on any input vector of choice by feeding 
the vector {\tt predict\_x} to the 
network, which outputs an array {\tt extrapolation}.

\end{itemize}

The datafiles that are stored upon successfully
executing this script are the following:

\begin{itemize}

\item {\tt Prediction\_k} contains the energy loss {\tt train\_x}, the MC training data {\tt train\_y}
and the ZLP prediction made on the array {\tt train\_x}, where {\tt k} is the $k$-th replica. 
\item {\tt Cost\_k} contains the training and validation error for the
  $k$-th replica
stored after each display step. 
The minimum of the the validation array is used to restore the optimal
neural network parameters.
\item {\tt Extrapolation\_k} contains the arrays {\tt predict\_x} and the ZLP predictions made on these values. 
\end{itemize}
These text files can be retrieved later to make new ZLP predictions
without the need to repeat the training procedure.
%
Futher, we store the optimal network parameters after each training session in the folder
'Models/Best\_models'. 
%
These can be loaded at a later stage
to make predictions for an arbitrary set of input variables. 

Running the loop over all replicas in series, using an input array of $\sim$50 training points 
and a total number of training epochs of 25000 per session,
takes approximately 20 seconds per optimization ($\sim$200 replicas per hour).





\noindent
\paragraph{\tt predictions.ipynb}
This script is used to analyse the predictions from the trained
neural networks that have been stored in the text files indicated above.

\begin{itemize}
  
\item {\it Import libraries and spectral data} from the {\bf load\_data.py} script.

\item {\it Create dataframes with all individual spectra.}
In order to later subtract all the predictions from the original individual spectra, we create a datafile
{\tt original} which contains the intensity values for each of the original input spectra restricted to the region between
 {\tt E\_min} and {\tt E\_max}.

\item {\it Load result files.}
  %
In order to import the files that were stored during the NN training, 
one should input to this script the right directions to find the prediction .txt files
by adjusting the lines {\tt path\_to\_data} and {\tt path\_predict}, {\tt path\_cost} and {\tt path\_extrapolate}, 
containing the predictions, cost function data and the extrapolation predictions respectively.

\item {\it Post-selection criteria.}
  %
  Here one select the datafiles that satisfy suitable post-fit selection
  criteria, such as the final error function being smaller
  than a certain threshold. 
 %
Once these datasets have been selected and stored in an array called {\tt use\_files},
we move on to the evaluation of the ZLP predictions. 

\item {\it Subtraction.}
 At this step the code uses the function {\tt matching()} to
  implement the matching procedure
  described in Sect.~\ref{sec:results_sample}.
  %
  It also automatically selects the
  values of $\Delta E_{\rm I}$ and $\Delta E_{\rm II}$ for the training session.
  %
  If the user aims to extract the bandgap properties
  from the onset of $I_{\rm inel}$, 
  the  {\tt bandgap()} function can be used to
  fit Eq.~\ref{eq:I1} to the onset region.

Here the code loops over the $N_{\rm rep}$ replicas and reads each prediction from the extrapolation data file {\tt predict\_x}.
%
For each replica {\tt k}, the code creates a datafile containing the original spectra intensities 
({\tt original['x*']} and {\tt original['y*']}), the predicted ZLP for this replica ({\tt prediction y}) 
and the predicted ZLP after matching with each spectrum ({\tt match *}). 
%
For each replica we subtract the matched spectrum from the original spectrum 
to obtain the desired subtraction: {\tt dif * = original * - match *}. 
%
This is done for each of the total of the replicas and all these results are stored in the  {\tt total\_replicas} dataframe. 
%
This file is saved in `Data/results/replica\_files' such that a user
can retrieve them  at any time to calculate the
statistical estimators such as prediction means and uncertainties. 

\item {\it Evaluate the subtracted spectra.}
%
Here the code creates a {\tt mean\_rep} file that contains
all the median predictions and the upper and lower bounds of the 68\% confidence intervals for 
the predicted ZLP, matched spectra and the subtracted spectra, for each of the original recorded
spectra originally given as an input. 
%
A graphical representation
of the result is then produced, showing the original spectrum, the matched
ZLP and the ZLP-subtracted spectrum including uncertainty bounds. 

\end{itemize}

We emphasize that the {\tt predictions\_pretrained\_net.ipynb} script is similar to the 
{\tt predictions.ipynb} script, but 
can be executed stand-alone
without the need to train again the neural networks, provided that
the model parameters corresponding to some previous training with
the desired input settings are available.
The item {\bf load result files} is now replaced by {\bf create result files}, 
which can be done by importing the pre-trained nets from the {\tt Models} folder. 



















