\section{Transition metal dichalcogenides}
\label{sec:tmd}

Over the past few years the exploration of 2D layered materials
has developed rapidly. 
%
In particular, significant attention has been 
going to monolayers of transition metal dichalcogenides (TMDs),
atomically thin semiconductor of the type MX$_2$, where M is a 
transition metal atom and X a chalcogen atom. 
%
The electronic structure of TMDs strongly depends on the coordination 
of the transition metal atoms, giving rise to an array of electronic
and magnetic properties~\cite{Chhowalla:2013}.
%
TMDs exhibit the interesting property to have an indirect
band gap in bulk form, whereas in monolayer form the gap becomes
direct~\cite{Splendiani:2010}.
%
The tunable electronic structure and the potential applications in
electronics makes these materials highly attractive for research. 
%
The indirect-to-direct bandgap transition is manifested as enhanced
photoluminescence in monolayers of, among others, tungsten disulfide
(WS$_2$), whereas only little emission is observed in bulk form.
%
WS$_2$ adopts a layered structure isotopic with MoS$_2$ and is studied
for potential applications such as storage of hydrogen and lithium~\cite{Bhandavat:2012}.
%

To be continued: polytypism

