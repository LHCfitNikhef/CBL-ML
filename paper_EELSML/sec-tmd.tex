\section{Transition Metal Dichalcogenides}
\label{sec:tmd}

Two-dimensional (2D) materials, also known as van der Waals (vdW) or layered materials,
are characterised by the remarkable property of being fully functional down to a single atomic layer.
%
Perhaps the most renowned member of this family of materials is graphene,
which benefits from many interesting properties such as efficient heat and electricity
conduction.
%
However, under normal conditions graphene lacks of a band gap, hampering its potential
applications in nano-electronics.
%
Recently, significant attention has been devoted to other types
of two-dimensional materials known as  transition metal dichalcogenides (TMDs).
%
These materials are of the form MX$_2$, where M is a 
transition metal atom (such as Mo or W) and X a chalcogen atom (such as S, Se, or Te). 
%
The crystalline structure of TMDs is such that
one layer of M atoms is sandwiched between two layers of X atoms.

The electronic structure of TMDs strongly depends on the coordination 
of the transition metal atoms, giving rise to an array of electronic
and magnetic properties~\cite{Chhowalla:2013}.
%
Further, the properties of this class of materials vary quite significantly
with their thickness, for instance MoS$_2$ exhibits an indirect bandgap
in the bulk form while it becomes direct at the monolayer level~\cite{Splendiani:2010}.
%
Such a tunable electronic structure and the potential applications in
nano-electronics makes these materials highly attractive for research. 

In this work we are interested in studying of the local electronic
properties of tungsten disulfide, WS$_2$, specifically of the
WS$_2$ nanostructures presented in~\cite{SabryaWS2}.
%
This material also exhibits an indirect-to-direct bandgap transition when going
from bulk to bilayer or monolayer form.
%
The effects of this transition are manifested as enhanced
photoluminescence in monolayer WS$_2$, whereas only little emission is observed in
the corresponding bulk form.
%
Further applications of this material include storage of hydrogen 
and lithium~\cite{Bhandavat:2012} for batteries.

As for other TMD materials, WS$_2$ adopts a layered structure 
by stacking atomic layers of S-W-S in a sandwich-like configuration. 
%
Although the interaction between adjacent layers is a weak Van der Waals 
force, the dependence of the interlayer interaction on the stacking 
order of WS$_2$ is significant.
%
Therefore, modulating the electronic
structure in a well-controlled way is crucial for application to
nano devices.
%
As discussed in~\cite{SabryaWS2},
a phenomenon called polytypism is an important factor that determines the interlayer
interactions within WS$_2$: different stacking types tend to coexist, 
complicating the characterization of the physical properties~\cite{Na:2018}.
%
One response of different stacking patterns to electric fields is
spontaneous electrical polarization, leading to modifications on the 
electronic band structure and correspondingly on the band gap~\cite{Li:2016}.

In this work we will analyse EEL spectra taken on WS$_2$ flower-like nanostructures~\cite{SabryaWS2}.
%
These nanoflowers  exhibit a  variety of morphologies, including lying flakes (“petals”) and
edge-exposed standing petals, arising from a common point (the “stem”).
%
 Structural analysis reveals that they exhibit a mixture of the 2H and 3R crystalline phases. 
%
EELS measurements were used to reveal the nature of the surface and edge excitations in these WS2 nanoflowers.









