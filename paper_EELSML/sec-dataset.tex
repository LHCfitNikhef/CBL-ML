%%%%%%%%%%%%%%%%%%%%
\section{Dataset}
%%%%%%%%%%%%%%%%%%%%
\label{sec:dataset}

In this section we present the experimental data used
in the present analysis.
%
It is divided into EEL spectra taken in vacuum, for calibration
purposes, and spectra taken in sample, for the physics interpretation.
%
As it is well known, the ZLP will in general be different
in the two cases: for in-sample spectra, one expects a broadening
arising from elastic electron-sample interactions that is missing in
the vacuum spectra.

\paragraph{Vacuum spectra}
%
For the ML model of the ZLP, four different sets of vacuum recordings
have been used, each measured under a different setting of exposure time 
and beam current. 
%
The data sets were recorded with a Titan TEM, equipped with a Skottky field emitter.
A maximum energy resolution (FWHM) of 0.003 eV could be realized. 
%
Table~\ref{table:vacuumdata} below indicates for each of the data sets the number of data files, 
the energy loss range, maximum recorded intensity and FWHM. 
%
These four data sets have been used to construct a ML-based
multidimensional model of the ZLP which can be inter- and extrapolated
to other operation conditions of the microscope.

%%%%%%%%%%%%%%%%%%%%%%%%%%%%%%%%%%%%%%%%%%%%%%%%%%%%%%%%%%%%%%%%%%%%%


\begin{table}[h]
  \caption{Properties of the zero loss peaks acquired in vacuum, used as inputs for training the multidimensional neural network model.}
  \begin{tabular}{@{}llllllllll}
\br
Set & $t_{\rm exp}$ {(}ms{)} & $E_{\rm beam}$ {(}keV{)} & $N_{\rm files}$ & $N_{dat} / file$ & $\Delta E_{\rm min}$  & $\Delta E_{\rm max}$  & $I_{\rm max}$ & FWHM  \\ 
\mr
1        & 100                 & 200                  & 15          & 2048               & -0.96              & +8.51               & 739770       & 0.025         \\
2        & 100                 & 60                   & 7           & 2048               & -0.54              & +5.59               & 326483       & 0.022         \\
3        & 10                  & 200                  & 12          & 2048               & -0.18              & +2.97               & 70913        & 0.003         \\
4        & 10                  & 60                   & 6           & 2048               & -0.40              & +4.78               & 30793        & 0.017         \\ 
\br
\end{tabular}
\label{table:vacuumdata}
\end{table}
%%%%%%%%%%%%%%%%%%%%%%%%%%%%%%%%%%%%%%%%%%%%%%%%%%%%%%%%5

\paragraph{Sample spectra}
%
In this work we will study the local electronic properties
via a low-loss region analysis of EELS taken on
WS$_2$ nanostructures.
%
WS$_2$ is a highly promising material exhibiting a wide range of 
possible applications for electronic and optical devices.
%
When WS$_2$ is thinned down to a single monolayer, its 
indirect band gap switches to a direct band gap of around 1.35 eV.
%
A collection of electron loss spectra acquired at different positions 
at the specimen is used to construct the neural network training inputs. 
%
These sets of data were obtained directly from the work of (... ref to Sonia).
%
A specimen image of the positions can be observed in figure~\ref{fig:ws2positions}.  
This nanostructure exhibits flat layers with different thicknesses, which 
can be distinguished from the picture as color differences.
%
Energy loss spectra obtained at positions 1-3 are vacuum recordings, 
positions 4-13 represent in-sample data.


