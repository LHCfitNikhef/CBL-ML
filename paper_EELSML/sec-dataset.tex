%%%%%%%%%%%%%%%%%%%%
\section{Dataset}
%%%%%%%%%%%%%%%%%%%%
\label{sec:dataset}

In this section we present the experimental data used
in the present analysis.
%
It is divided into EEL spectra taken in vacuum, for calibration
purposes, and spectra taken in sample, for the physics interpretation.
%
As it is well known, the ZLP will in general be different
in the two cases: for in-sample spectra, one expects a broadening
arising from elastic electron-sample interactions that is missing in
the vacuum spectra.
%
These datasets will be used as input for the training of our
machine learning model of the ZLP, to be described in the next section.

\paragraph{Vacuum spectra.}
%
For the purpose of this study, four different sets of vacuum recorded zero loss peaks are used, each of them measured under a different setting of exposure time and beam current, as listed in Table~\ref{table:vacuum}. These data sets were recorded with a Titan TEM, equipped with a Skottky field emitter.
%
In each case
indicate
the number of data points ($N_{\rm files}\times N_{dat}$/file), the energetic range, maximum recorded intensity and FWHM of each set of data. FWHM, $dE_{\rm min}$ and $dE_{\rm max}$ are given in eV.
%
The data from Table~\ref{table:vacuum} will be use to construct a ML-based
multidimensional model of the ZLP which can be inter- and extrapolated
to other operation conditions of the microscope.

%%%%%%%%%%%%%%%%%%%%%%%%%%%%%%%%%%%%%%%%%%%%%%%%%%%%%%%%%%%%%%%%%%%%%
\begin{table}[h]
  \renewcommand{\arraystretch}{1.40}
\begin{tabular}{|l|l|l|l|l|l|l|l|l|}
\toprule
Set & $t_{\rm exp}$ {[}ms{]} & $E_{\rm beam}$ {[}keV{]} & $N_{\rm files}$ & $N_{dat} / file$ & $dE_{\rm min}$  & $dE_{\rm max}$  & $I_{\rm max}$ & FWHM  \\ \hline
1        & 100                 & 200                  & 15          & 2048               & -0.96              & +8.51               & 739770       & 0.025         \\
2        & 100                 & 60                   & 7           & 2048               & -0.54              & +5.59               & 326483       & 0.022         \\
3        & 10                  & 200                  & 12          & 2048               & -0.18              & +2.97               & 70913        & 0.003         \\
4        & 10                  & 60                   & 6           & 2048               & -0.40              & +4.78               & 30793        & 0.017         \\ \bottomrule
\end{tabular}
\vspace{0.4cm}
\caption{The data sets of the in vacuum recorded zero loss peaks used for this part of the analysis. We show the number of data points ($N_{files}\times N_{dat}$/file), the energetic range, maximum recorded intensity and FWHM of each set of data. FWHM, $dE_{min}$ and $dE_{max}$ are given in eV.}
\label{table:vacuum}
\end{table}
%%%%%%%%%%%%%%%%%%%%%%%%%%%%%%%%%%%%%%%%%%%%%%%%%%%%%%%%5


\paragraph{Sample spectra}
%
In this work we will study the local electronic properties
via a low-loss region analysis of EELS taken on
MoS$_2$ nanostructures~ \cite{soniamos2}.
%
MoS$_2$ is a highly promising material exhibiting a wide range of possible applications for electronic and optical devices.
%
When MoS$_2$ is thinned down to a single monolayer (ML), its indirect band gap switches to a direct band gap of around 1.88 eV~\cite{Nerl:2016}.

To obtain these measurements,
a monochromated electron source was used operating at 60 kV, which achieves an energy resolution of around 30 meV. The most-distinctive features of the EELS spectra are two relatively narrow peaks around 1.88 and 2.05 eV, arising from the direct exciton transitions and split by interlayer interactions and spin-orbit coupling. These measurements for the position of the bandgap is consistent with previous studies~\cite{Nerl:2016, Komsa:2012}. The peaks could be revealed after subtraction of a fitted Gaussian curve and a power law to the ZLP and its right-hand tail. 
%
An ensemble of electron loss spectra acquired at different positions at the sample is used to construct the training inputs (Fig. \ref{spectra}). As the ZLP is several orders of magnitude bigger than the sample data, the logarithm of the intensity is used to train the model.
%
The main features of our data sets are summarized in Table \ref{table:spectra}, where we show (...). 

\begin{table}[H]
\centering
\begin{tabular}{|l|l|}
\hline
Property of dataset & Values \\ \hline
1                   & x      \\ \hline
2                   & x      \\ \hline
3                   & x      \\ \hline
4                   & x      \\ \hline
5                   & x      \\ \hline
\end{tabular}
\caption{Table 1}
\label{table:spectra}
\end{table}
