\section{Introduction}
\label{sec:introduction}

Electron energy-loss spectroscopy (EELS) within the transmission electron microscope (TEM) provides
a wide range of
valuable information on the structural, chemical, and electronic properties of nanoscale materials.
%
Thanks to recent instrumentation breakthroughs
such as electron monochromators~\cite{Terauchi:2005, Freitag:2005} and aberration correctors~\cite{Haider:1998},
modern EELS analyses can map these properties with unprecedented spatial and spectral resolution.
%
A particularly important region of the EEL spectra is
the low-loss region, defined by those electrons that have lost
less than a few eV ($\Delta E\lsim 5$ GeV) following their inelastic interactions
with the sample.
%
The analysis of this low-loss region makes possible charting the local
electronic properties of nanomaterials~\cite{Geiger:1967}, from the characterisation of
bulk and surface plasmons~\cite{Schaffer:2008}, excitons~\cite{Erni:2005}, 
inter- and intra-band transitions~\cite{Rafferty:1998},
and phonons to the determination of their bandgap and band structure~\cite{Stoger:2008}.

Provided the sample is electron-transparent, as required for TEM inspection,
in EELS the bulk of the incident electron beam will traverse it
either without interacting or restricted to elastic scatterings with the atoms
of the sample's crystalline lattice.
%
These electrons are recorded as a narrow,
 high intensity peak centered at energy losses
of $\Delta E\simeq 0$ and known as the zero loss peak (ZLP).
%
The energy resolution of EELS analyses is ultimately determined by
the electron beam size of the system, often expressed in terms
of the full width at half maximum (FWHM) of the
ZLP~\cite{Egerton:2009}.
%
In the low-loss region, the contribution from the ZLP
often overwhelms that from  from the inelastic scatterings arising with
the interactions of the beam electrons
with the sample.
%
Therefore, relevant signals of low-loss phenomena such as excitons,
phonons, and intraband transitions risk being drowned
in ZLP tail~\cite{Abajo:2010}.
%
An accurate removal of the ZLP
contribution is thus crucial  in order to efficiently chart and identify the  features
of the low-loss  region in EEL spectra. 

The properties of the ZLP in monochromated EELS depend on the electron energy dispersion,
the monochromator alignment, and the sample thickness~\cite{Park:2008, Stoger:2008}.
%
The first two factors arise already in the absence of a specimen (vacuum operation),
while the third one is associated
to elastic scatterings with the sample such as atomic scatterings,
phonon excitation, and exciton losses.
%
This implies that  measurements of vacuum EEL spectra can be used for calibration purposes
but not to subtract the ZLP from spectra taken on specimens, since their shapes will differ
in general.

Several approaches to ZLP subtraction\cite{Rafferty:2000, Stoger:2008, Egerton:1996} have been put forward in literature.
%
These are often based on specific model assumptions about the ZLP properties, specifically
concerning its parametric functional dependence on the electron energy loss $\Delta E$,
from Lorentzian~\cite{Dorneich:1998}
and power laws~\cite{Erni:2005} to more general multiple-parameter functions~\cite{Benthem:2001}.
%
Another approach is based on the mirroring the $\Delta E <0$ region of the spectra, assuming
that the $\Delta E>0$ region is fully symmetric~\cite{Lazar:2003}.
%
More recent studies use integrated software applications for background subtraction 
methods~\cite{Egerton:10.1016/S0304-3991(01)00155-3, Held:2020, Granerod:2018, Fung:2020}.
%
These  subtraction methods are however affected by three main limitations.
%
Firstly, they rely on specific model assumptions {\it e.g.} with
the choice of fir function, introducing a methodological
bias whose size is difficult to quantify.
%
Secondly, they lack an estimate of the associated uncertainties, which in turn affects
the reliability of any physical interpretations of the low loss region such as
band gap extraction.
%
Thirdly, {\it ad hoc} choices of such as those of the fitting ranges introduce a significant degree of
arbitrariness in the procedure.

Here we bypass these limitations by developing a model-independent strategy
that makes possible a multidimensional determination of the ZLP
with a faithful uncertainty estimate.
%
Our approach is based on machine learning (ML) techniques
developed in high-energy physics to study
quark and gluon substructure of protons
particle collisions~\cite{Ball:2008by,Ball:2012cx,Ball:2014uwa,Ball:2017nwa}.
%
It is based on the  Monte Carlo replica  method to construct a probability
distribution in the space of experimental data and artificial
neural networks as unbiased interpolators to parametrise the ZLP.
%
The result is
a faithful sampling of the probability distribution in the space of ZLP
which can be used to subtract its contribution to EEL spectra while
propagating the associated uncertainties.
%
Further, we can extrapolate this ZLP parametrisation to other TEM
operation conditions beyond those used in the training dataset.

Our work is divided into two main parts.
%
In the first one, we construct a ML model of ZLP spectra taken
in vacuum able to accommodate an  arbitrary number of input
variables corresponding different  operation conditions of the TEM.
%
We demonstrate how the model describes successfully the
input spectra and assess its extrapolation for other operation
conditions.
%
In the second part, we construct a one-dimensional model
of the ZLP as a function of $\Delta E$ from spectra acquired on
tungsten disulfide (WS$_2$) nanoflowers~\cite{SabryaWS2}.
%
The resulting subtracted spectra are used to determine
the value and type of the WS$_2$ bandgap
and its dependence on the underlying crystalline morphology of these nanostructures.

The paper is organized as follows.
%
First of all, in Sect.~\ref{sec:eels}
we review the main features of the EELS technique
and its application to nanostructres built upon transition metal
dichalcogenide (TMD) materials, with emphasis on WS$_2$.
%
In Sect.~\ref{sec:methodology} we describe our machine learning methodology
for the ZLP parametrisation.
%
Sects.~\ref{sec:results_vacuum} and~\ref{sec:results_sample} contain
our results for the ZLP parametrisation for spectra acquired
in vacuum and in sample respectively, which in the latter
case allows us to probe the local band structure properties
of the WS$_2$ nanoflowers.
%
Finally in Sect.~\ref{sec:summary} we summarise
and outline possible future developments.
%
Our results have been obtained with an open-source {\sc Python} code,
{\tt EELfitter}, presented in App.~\ref{sec:installation}
together with some installation and usage instructions.
