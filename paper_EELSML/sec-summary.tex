%%%%%%%%%%%%%%%%%%%%%%%%%%%%%%%%%%%%%%%%%
\section{Summary and outlook}
%%%%%%%%%%%%%%%%%%%%%%%%%%%%%%%%%%%%%%%%%
\label{sec:summary}

In this work we have presented a novel strategy to parametrise
the zero-loss peak present in electron energy loss spectroscopy
by applying machine learning techniques developed
in the context of high-energy physics.
%
We have demonstrated how, in the case of vacuum spectra, the model
is flexible enough to accomodate several input variables without
imposing any {\it ad-hoc} functional form.
%
Once applied to the parametrization of the ZLP in EEL spectra taken
on samples, our approach allows for a robust subtraction of its contribution
from the very low-loss region of the spectra with a faithful
estimate of all relevant uncertainties.

As a proof of concept we have applied our novel ZLP subtraction
strategy to spectra recorded in different regions of WS$_2$ nanostructures.
%
The subtracted spectra have then been used to extract the value
of the bandgap $E_{\rm bg}$ and to estimate the type of bandgap,
either direct or indirect.
%
Our findings demonstrate that ....

The approach presented in this work could be extended
in several directions.
%
To begin with, it could be applied to study the bandgap
and local electronic properties of other types
of nanostructures built upon layered materials.
%
It would also be interested to  further exploit the subtracted EEL spectra
in order to determine the complex dielectric function and its associated
uncertainties by means of the Kramers-Kronig relation.
%
Another possible application would be the study of spectral TEM images,
where each pixel in the image contains an individual EEL spectrum.
%
Here machine learning methods would also be useful in order
to  automatically identify relevant features (peaks, edges, shoulders)
without having to process each spectrum individually and assess
how these features vary as we move along different regions of the
nanostructure.

\subsection*{Acknowledgments}
%
The work of J.~R. has been partially supported by the
Dutch Research Council (NWO).
%
We are grateful to Emanuele R. Nocera and Jake Ethier for
assistance in installing the code in the Nikhef cluster.
