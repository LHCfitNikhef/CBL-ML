%%%%%%%%%%%%%%%%%%%%%%%%%%%%%%%%%%%%%%%%%
\section{Summary and outlook}
%%%%%%%%%%%%%%%%%%%%%%%%%%%%%%%%%%%%%%%%%
\label{sec:summary}

In this work we have presented a novel strategy to parametrise and remove
the zero-loss peak that appears in the low-loss region
of electron energy loss spectroscopy measurements.
%
Our strategy is based on machine learning techniques developed
in the context of high-energy physics, in particular for studies of the quark
and gluon substructure of protons.
%
An important advantage of this method is the faithful estimation of the
input data and methodological uncertainties and their propagation to the subtracted
spectra without the need to rely on any kind of approximation.
%
We have demonstrated how, in the case of vacuum spectra, the model
is flexible enough to accomodate several input variables, corresponding
to different operation conditions,  without the need to
assume any {\it ad-hoc} functional form for the ZLP parametrisation.
%
This flexibility allows us for example to extrapolate the expected FWHM of the ZLP
corresponding to other operation conditions of the microscope, beyond those
included in the training set.
%
When applied to EEL spectra recorded on samples, we have shown
that our approach allows to cleanly disentangle the ZLP contribution from
that of the inelastic emissions from the sample, and produce
subtracted spectra  that account for all relevant sources of uncertainty.
%
This way it becomes possible to fully exploit
the valuable physical information contained in the ultra low-loss region of
these spectra.

As a proof of concept we have applied this ZLP subtraction
strategy to EEL spectra recorded in thick regions of WS$_2$ nanoflowers, 
where WS$_2$ is expected to behave as a bulk material.
%
We used the results to estimate the local value of the bandgap energy $E_{\rm BG}$
and to assess whether this bandgap is direct or indirect.
%
The final values for $E_{\rm BG}$ and $b$ obtained in the analysis of this specific spectrum are
E_{\rm BG} = 1.6_{-0.2}^{+0.3}\,{\rm eV} \, ,\quad b= 1.3_{-0.7}^{+0.3} \, .
%
Our analysis displays preference for an indirect bandgap of approximately 1.6 eV,
which is consistent with 
theoretical expectations of bulk WS$_2$.
%
As a second proof of concept, we applied our methods to a much thinner sample of WS$_2$, 
composed of overlapping petals whose thicknesses can be as small as a few monolayers.
%
Instead of studying the bandgap properties, one could exploit the ZLP-subtracted results 
of this sample to study the local
excitonic transitions that are observed in the ultra-low-loss region of the spectra.
%
The ZLP-subtracted spectra in this sample have allowed
the charting of exciton peaks down to $\Delta E\simeq 1.5$ eV together with
the associated uncertainty estimate, allowing to establish their significance.
%
\newparagraph
The approach presented in this work could be extended
in several directions.
%
First of all, it would be interesting to test its robustness when additional
operation conditions of the microscope are included as input variables,
and to assess to which extend ZLP models obtained with an specific microscope
can be generalised to an altogether different TEM.
%
Further, a strong cross-check of our method would be provided by comparing
our predictions for other operation conditions of the microscope, such
as the FWHM as a function of the beam energy $E_b$ reported in Fig.~\ref{fig:extrapolbeam}
with actual measurements and verifying whether or not there is agreement within the
uncertainties of the prediction.
%
Concerning the physical interpretation of the low-loss region of EEL
spectra, our method could be applied to study the bandgap
and other local electronic properties of different types
of nanostructures built upon 2D layered materials, such as MoS$_2$ nanowalls
and nano-pillars and WS$_2$/MoS$_2$ arrays or heterostructures.
%
In addition to the bandgap characterisation, one might
consider the implications of our approach for the study
of other phenomena relevant for the interpretation of the low-loss
region such as  plasmons, excitons, phonon interactions, and
intra-band transitions.
%
One could   further exploit the subtracted EEL spectra produced
with our method to evaluate the complex dielectric function and its associated
uncertainties by means of the Kramers-Kronig relation.
%
Such phenomenological studies of the local electronic properties would be compared
with {\it ab initio} calculations such as Density Functional Theory, based
on the same underlying crystalline structure of the analysed samples.
%
We recall that the results
presented in this work are to the best of our knowledge the first EELS bandgap
analysis of WS$_2$ nanostructures based on mixed 2H/3R polytypes.

Another possible generalisation of our method would be to the study of spectral TEM images,
where each pixel in the image contains an individual EEL spectrum (possibly
extended to 3D images).
%
Here machine learning methods would be useful in order
to  identify relevant features of the spectra (peaks, edges, shoulders) in a fully
automated way
without having to process each spectrum individually, and then assess
how these features vary as we move along different regions of the
nanostructure.
%
Such application would combine two of the most topical uses of machine learning, regression
on the one hand and classification on the other hand.

\subsection*{Acknowledgments}
%
The work of J.~R. has been partially supported by the
Dutch Research Council (NWO).
%
We are grateful to Emanuele R. Nocera and Jake Ethier for
assistance in installing the code in the Nikhef cluster.
