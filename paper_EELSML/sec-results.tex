\section{Results}
\label{sec:results}

In this section we present the main results of this work.


\subsection{Band-gap determination}
Traditionally, bandgaps of dielectrics have been measured by methods offering high energy resolution, however limited spatial resolution \cite{Park:2008}. Attaching electron monochromators to transmission electron microscopes has proven to be a powerful method for obtaining high spatial resolution, although still limited by the beam size and delocalization phenomena. 

As suggested in literature \cite{15,14}, the threshold for bandgap detection is the Kimoto limit, which is the energy loss at which 1/1000th of the maximum peak intensity is measured: at this level, sample contributions can be detected above the background of the tails of the zero-loss distribution. We should make sure that we are working above this limit, which for our dataset corresponds to (...) eV.\\

Once the subtracted spectra have been obtained, one can estimate the bandgap, to a first approximation, as the inflection point of the rising intensity. The value can also be roughly estimated from the onset of the absorption or from a linear fit to the maximum positive slope in the EELS spectrum \cite{Schamm:2003}. However, a more accurate and reliable determination is based on the work of Rafferty and Brown  \cite{Rafferty:2000}. The onset of the subtracted spectrum for a a meterial with a direct bandgap is expected to follow a function of the type
\begin{equation} \label{eq:I1}
    I(E) = I_0 + c\cdot(dE-E_{BG})^{(1/2)}
\end{equation}
where $I_0$ and c are constants, $dE$ is the energy loss and $E_{BG}$ is the bandgap energy. For an indirect bandgap, the power of $(1/2)$ changes to $(3/2)$. Therefore, the bandgap nature (direct or indirect) can be extracted by least-squares fitting of each $(k)$-th replica subtracted spectrum to equation \ref{eq:I3}:
\begin{equation} \label{eq:I3}
    I(E) = I_0 + c\cdot(dE-E_{BG})^{(b)}.
\end{equation}
This way, by summation over all replicas $N_{rep}$, we can determine the expectation value $\textless{b}\textgreater{}$ and its uncertainty $\sigma_b$. One has to take into account that this approach is very sensitive to the subtraction procedure: the free parameter $E_{BG}$ is the onset of the subtracted spectrum intensity, and it should be investigated how the choice of $dE_1$ affects the free parameter $E_{BG}$.
